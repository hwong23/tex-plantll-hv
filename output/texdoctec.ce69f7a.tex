%%
% Copyright (c) 2017 - 2024, Pascal Wagler;
% Copyright (c) 2014 - 2024, John MacFarlane
%
% All rights reserved.
%
% Redistribution and use in source and binary forms, with or without
% modification, are permitted provided that the following conditions
% are met:
%
% - Redistributions of source code must retain the above copyright
% notice, this list of conditions and the following disclaimer.
%
% - Redistributions in binary form must reproduce the above copyright
% notice, this list of conditions and the following disclaimer in the
% documentation and/or other materials provided with the distribution.
%
% - Neither the name of John MacFarlane nor the names of other
% contributors may be used to endorse or promote products derived
% from this software without specific prior written permission.
%
% THIS SOFTWARE IS PROVIDED BY THE COPYRIGHT HOLDERS AND CONTRIBUTORS
% "AS IS" AND ANY EXPRESS OR IMPLIED WARRANTIES, INCLUDING, BUT NOT
% LIMITED TO, THE IMPLIED WARRANTIES OF MERCHANTABILITY AND FITNESS
% FOR A PARTICULAR PURPOSE ARE DISCLAIMED. IN NO EVENT SHALL THE
% COPYRIGHT OWNER OR CONTRIBUTORS BE LIABLE FOR ANY DIRECT, INDIRECT,
% INCIDENTAL, SPECIAL, EXEMPLARY, OR CONSEQUENTIAL DAMAGES (INCLUDING,
% BUT NOT LIMITED TO, PROCUREMENT OF SUBSTITUTE GOODS OR SERVICES;
% LOSS OF USE, DATA, OR PROFITS; OR BUSINESS INTERRUPTION) HOWEVER
% CAUSED AND ON ANY THEORY OF LIABILITY, WHETHER IN CONTRACT, STRICT
% LIABILITY, OR TORT (INCLUDING NEGLIGENCE OR OTHERWISE) ARISING IN
% ANY WAY OUT OF THE USE OF THIS SOFTWARE, EVEN IF ADVISED OF THE
% POSSIBILITY OF SUCH DAMAGE.
%%

%%
% This is the Eisvogel pandoc LaTeX template.
%
% For usage information and examples visit the official GitHub page:
% https://github.com/Wandmalfarbe/pandoc-latex-template
%%

% Options for packages loaded elsewhere
\PassOptionsToPackage{unicode}{hyperref}
\PassOptionsToPackage{hyphens}{url}
\PassOptionsToPackage{dvipsnames,svgnames,x11names,table}{xcolor}
%
\documentclass[
  paper=a4,
  ,captions=tableheading
]{scrartcl}
\usepackage{amsmath,amssymb}
% Use setspace anyway because we change the default line spacing.
% The spacing is changed early to affect the titlepage and the TOC.
\usepackage{setspace}
\setstretch{1.2}
\usepackage{iftex}
\ifPDFTeX
  \usepackage[T1]{fontenc}
  \usepackage[utf8]{inputenc}
  \usepackage{textcomp} % provide euro and other symbols
\else % if luatex or xetex
  \usepackage{unicode-math} % this also loads fontspec
  \defaultfontfeatures{Scale=MatchLowercase}
  \defaultfontfeatures[\rmfamily]{Ligatures=TeX,Scale=1}
\fi
\usepackage{lmodern}
\ifPDFTeX\else
  % xetex/luatex font selection
\fi
% Use upquote if available, for straight quotes in verbatim environments
\IfFileExists{upquote.sty}{\usepackage{upquote}}{}
\IfFileExists{microtype.sty}{% use microtype if available
  \usepackage[]{microtype}
  \UseMicrotypeSet[protrusion]{basicmath} % disable protrusion for tt fonts
}{}
\makeatletter
\@ifundefined{KOMAClassName}{% if non-KOMA class
  \IfFileExists{parskip.sty}{%
    \usepackage{parskip}
  }{% else
    \setlength{\parindent}{0pt}
    \setlength{\parskip}{6pt plus 2pt minus 1pt}}
}{% if KOMA class
  \KOMAoptions{parskip=half}}
\makeatother
\usepackage{xcolor}
\definecolor{default-linkcolor}{HTML}{A50000}
\definecolor{default-filecolor}{HTML}{A50000}
\definecolor{default-citecolor}{HTML}{4077C0}
\definecolor{default-urlcolor}{HTML}{4077C0}
\usepackage[top=1.3in,bottom=1in]{geometry}
\usepackage[export]{adjustbox}
\usepackage{graphicx}
\usepackage{longtable,booktabs,array}
\usepackage{calc} % for calculating minipage widths
% Correct order of tables after \paragraph or \subparagraph
\usepackage{etoolbox}
\makeatletter
\patchcmd\longtable{\par}{\if@noskipsec\mbox{}\fi\par}{}{}
\makeatother
% Allow footnotes in longtable head/foot
\IfFileExists{footnotehyper.sty}{\usepackage{footnotehyper}}{\usepackage{footnote}}
\makesavenoteenv{longtable}
% add backlinks to footnote references, cf. https://tex.stackexchange.com/questions/302266/make-footnote-clickable-both-ways
\usepackage{footnotebackref}
\usepackage{graphicx}
\makeatletter
\newsavebox\pandoc@box
\newcommand*\pandocbounded[1]{% scales image to fit in text height/width
  \sbox\pandoc@box{#1}%
  \Gscale@div\@tempa{\textheight}{\dimexpr\ht\pandoc@box+\dp\pandoc@box\relax}%
  \Gscale@div\@tempb{\linewidth}{\wd\pandoc@box}%
  \ifdim\@tempb\p@<\@tempa\p@\let\@tempa\@tempb\fi% select the smaller of both
  \ifdim\@tempa\p@<\p@\scalebox{\@tempa}{\usebox\pandoc@box}%
  \else\usebox{\pandoc@box}%
  \fi%
}
% Set default figure placement to htbp
% Make use of float-package and set default placement for figures to H.
% The option H means 'PUT IT HERE' (as  opposed to the standard h option which means 'You may put it here if you like').
\usepackage{float}
\floatplacement{figure}{H}
\makeatother
\setlength{\emergencystretch}{3em} % prevent overfull lines
\providecommand{\tightlist}{%
  \setlength{\itemsep}{0pt}\setlength{\parskip}{0pt}}
\setcounter{secnumdepth}{-\maxdimen} % remove section numbering
\ifLuaTeX
\usepackage[bidi=basic]{babel}
\else
\usepackage[bidi=default]{babel}
\fi
\babelprovide[main,import]{english}
% get rid of language-specific shorthands (see #6817):
\let\LanguageShortHands\languageshorthands
\def\languageshorthands#1{}
\makeatletter
\@ifpackageloaded{subfig}{}{\usepackage{subfig}}
\@ifpackageloaded{caption}{}{\usepackage{caption}}
\captionsetup[subfloat]{margin=0.5em}
\AtBeginDocument{%
\renewcommand*\figurename{Figura}
\renewcommand*\tablename{Tabla}
}
\AtBeginDocument{%
\renewcommand*\listfigurename{Lista de Figuras}
\renewcommand*\listtablename{Lista de Tablas}
}
\newcounter{pandoccrossref@subfigures@footnote@counter}
\newenvironment{pandoccrossrefsubfigures}{%
\setcounter{pandoccrossref@subfigures@footnote@counter}{0}
\begin{figure}\centering%
\gdef\global@pandoccrossref@subfigures@footnotes{}%
\DeclareRobustCommand{\footnote}[1]{\footnotemark%
\stepcounter{pandoccrossref@subfigures@footnote@counter}%
\ifx\global@pandoccrossref@subfigures@footnotes\empty%
\gdef\global@pandoccrossref@subfigures@footnotes{{##1}}%
\else%
\g@addto@macro\global@pandoccrossref@subfigures@footnotes{, {##1}}%
\fi}}%
{\end{figure}%
\addtocounter{footnote}{-\value{pandoccrossref@subfigures@footnote@counter}}
\@for\f:=\global@pandoccrossref@subfigures@footnotes\do{\stepcounter{footnote}\footnotetext{\f}}%
\gdef\global@pandoccrossref@subfigures@footnotes{}}
\@ifpackageloaded{float}{}{\usepackage{float}}
\floatstyle{ruled}
\@ifundefined{c@chapter}{\newfloat{codelisting}{h}{lop}}{\newfloat{codelisting}{h}{lop}[chapter]}
\floatname{codelisting}{Listing}
\newcommand*\listoflistings{\listof{codelisting}{Listas del Documento}}
\makeatother
\usepackage{bookmark}
\IfFileExists{xurl.sty}{\usepackage{xurl}}{} % add URL line breaks if available
\urlstyle{same}
\hypersetup{
  pdftitle={Informe de Certificación Operativa Plataforma de Software Trii.co},
  pdfauthor={SoftwareProductiva.com},
  pdflang={en},
  pdfsubject={Implementación Proyecto},
  pdfkeywords={Rendimiento, Métodos pruebas, Pruebas software, QA},
  hidelinks,
  breaklinks=true,
  pdfcreator={LaTeX via pandoc with the Eisvogel template}}
\title{Informe de Certificación Operativa Plataforma de Software
Trii.co}
\usepackage{etoolbox}
\makeatletter
\providecommand{\subtitle}[1]{% add subtitle to \maketitle
  \apptocmd{\@title}{\par {\large #1 \par}}{}{}
}
\makeatother
\subtitle{Borrador}
\author{SoftwareProductiva.com}
\date{2025-01-20}



%%
%% added
%%


%
% for the background color of the title page
%
\usepackage{pagecolor}
\usepackage{afterpage}
\usepackage{tikz}

%
% break urls
%
\PassOptionsToPackage{hyphens}{url}

%
% When using babel or polyglossia with biblatex, loading csquotes is recommended
% to ensure that quoted texts are typeset according to the rules of your main language.
%
\usepackage{csquotes}

%
% captions
%
\definecolor{caption-color}{HTML}{777777}
\usepackage[font={stretch=1.2}, textfont={color=caption-color}, position=top, skip=4mm, labelfont=bf, singlelinecheck=false, justification=raggedright]{caption}
\setcapindent{0em}

%
% blockquote
%
\definecolor{blockquote-border}{RGB}{221,221,221}
\definecolor{blockquote-text}{RGB}{119,119,119}
\usepackage{mdframed}
\newmdenv[rightline=false,bottomline=false,topline=false,linewidth=3pt,linecolor=blockquote-border,skipabove=\parskip]{customblockquote}
\renewenvironment{quote}{\begin{customblockquote}\list{}{\rightmargin=0em\leftmargin=0em}%
\item\relax\color{blockquote-text}\ignorespaces}{\unskip\unskip\endlist\end{customblockquote}}

%
% Source Sans Pro as the default font family
% Source Code Pro for monospace text
%
% 'default' option sets the default
% font family to Source Sans Pro, not \sfdefault.
%
\ifnum 0\ifxetex 1\fi\ifluatex 1\fi=0 % if pdftex
    \usepackage[default]{sourcesanspro}
  \usepackage{sourcecodepro}
  \else % if not pdftex
    \usepackage[default]{sourcesanspro}
  \usepackage{sourcecodepro}

  % XeLaTeX specific adjustments for straight quotes: https://tex.stackexchange.com/a/354887
  % This issue is already fixed (see https://github.com/silkeh/latex-sourcecodepro/pull/5) but the
  % fix is still unreleased.
  % TODO: Remove this workaround when the new version of sourcecodepro is released on CTAN.
  \ifxetex
    \makeatletter
    \defaultfontfeatures[\ttfamily]
      { Numbers   = \sourcecodepro@figurestyle,
        Scale     = \SourceCodePro@scale,
        Extension = .otf }
    \setmonofont
      [ UprightFont    = *-\sourcecodepro@regstyle,
        ItalicFont     = *-\sourcecodepro@regstyle It,
        BoldFont       = *-\sourcecodepro@boldstyle,
        BoldItalicFont = *-\sourcecodepro@boldstyle It ]
      {SourceCodePro}
    \makeatother
  \fi
  \fi

%
% heading color
%
\definecolor{heading-color}{RGB}{40,40,40}
\addtokomafont{section}{\color{heading-color}}
% When using the classes report, scrreprt, book,
% scrbook or memoir, uncomment the following line.
%\addtokomafont{chapter}{\color{heading-color}}

%
% variables for title, author and date
%
\usepackage{titling}
\title{Informe de Certificación Operativa Plataforma de Software
Trii.co}
\author{SoftwareProductiva.com}
\date{2025-01-20}

%
% tables
%

\definecolor{table-row-color}{HTML}{F5F5F5}
\definecolor{table-rule-color}{HTML}{999999}

%\arrayrulecolor{black!40}
\arrayrulecolor{table-rule-color}     % color of \toprule, \midrule, \bottomrule
\setlength\heavyrulewidth{0.3ex}      % thickness of \toprule, \bottomrule
\renewcommand{\arraystretch}{1.3}     % spacing (padding)


%
% remove paragraph indentation
%
\setlength{\parindent}{0pt}
\setlength{\parskip}{6pt plus 2pt minus 1pt}
\setlength{\emergencystretch}{3em}  % prevent overfull lines

%
%
% Listings
%
%


%
% header and footer
%
\usepackage[headsepline,footsepline]{scrlayer-scrpage}

\newpairofpagestyles{eisvogel-header-footer}{
  \clearpairofpagestyles
  \ihead*{2025-01-20}
  \chead*{}
  \ohead*{Informe de Certificación Operativa Plataforma de Software
Trii.co}
  \ifoot*{SoftwareProductiva.com}
  \cfoot*{}
  \ofoot*{\thepage}
  \addtokomafont{pageheadfoot}{\upshape}
}
\pagestyle{eisvogel-header-footer}



%%
%% end added
%%

%% change the default with of the images using \setkeys{Gin}{width=.8\linewidth}
\setkeys{Gin}{width=1\linewidth}

\begin{document}

%%
%% begin titlepage
%%
\begin{titlepage}
\newgeometry{top=2cm, right=4cm, bottom=3cm, left=4cm}
\tikz[remember picture,overlay] \node[inner sep=0pt] at (current page.center){\includegraphics[width=\paperwidth,height=\paperheight]{include/background.pdf}};
\newcommand{\colorRule}[3][black]{\textcolor[HTML]{#1}{\rule{#2}{#3}}}
\begin{flushleft}
\noindent
\\[-1em]
\color[HTML]{5F5F5F}
\makebox[0pt][l]{\colorRule[360049]{1.3\textwidth}{4pt}}
\par
\noindent

% The titlepage with a background image has other text spacing and text size
{
  \setstretch{2}
  \vfill
  \vskip -8em
  \noindent {\huge \textbf{\textsf{Informe de Certificación Operativa
Plataforma de Software Trii.co}}}
    \vskip 1em
  {\Large \textsf{Borrador}}
    \vskip 2em
  \noindent {\Large \textsf{SoftwareProductiva.com} \vskip 0.6em \textsf{2025-01-20}}
  \vfill
}

\noindent
\includegraphics[width=30mm, left]{include/logo.jpg}

\end{flushleft}
\end{titlepage}
\restoregeometry
\pagenumbering{arabic}

%%
%% end titlepage
%%

% \maketitle


\section{Contenido}\label{sec:contenido}

\begin{itemize}
\tightlist
\item
  \hyperref[informaciuxf3n-del-documento]{Información del Documento}
\item
  \hyperref[informe-operativo-plataforma-de-software-trii.co]{Informe
  Operativo Plataforma de Software Trii.co}
\item
  \hyperref[evaluaciuxf3n-del-rendimiento]{Evaluación del Rendimiento}
\item
  \hyperref[resultados-y-conclusiones-del-informe-de-rendimiento]{Resultados
  y Conclusiones del Informe de Rendimiento}
\item
  \hyperref[anexos-tuxe9cnicos]{Anexos Técnicos}
\end{itemize}

\newpage

\section{Información del
Documento}\label{sec:informaciuxf3n-del-documento}

\subsection{Versión del Documento}\label{sec:versiuxf3n-del-documento}

\begin{quote}
\end{quote}

\subsection{Control de Cambios}\label{sec:control-de-cambios}

Historia de cambios de la propuesta.

Versión actual: 1.ce69f7a - Compilación para entrega - Sun, 19 Jan 2025
04:28:26 +0000

Versiones Anteriores

1.94ce655 - Compilación para entrega - Fri, 17 Jan 2025 20:52:52 +0000

1.156cafc - Compilación para entrega - Fri, 17 Jan 2025 20:42:54 +0000

1.44f5d95 - Compilación para entrega - Fri, 17 Jan 2025 18:30:44 +0000

1.198992c - Compilación para entrega - Fri, 17 Jan 2025 18:27:42 +0000

\subsubsection{Realizado Por}\label{sec:realizado-por}

H. Wong, ing.

\subsubsection{Revisado Por}\label{sec:revisado-por}

(revisor), Trii.co

\newpage

\section{Informe Operativo Plataforma de Software
Trii.co}\label{sec:informe-operativo-plataforma-de-software-trii.co}

\subsection{Componentes del Informe de Rendimiento y Capacidad de la
Plataforma
Trii.co}\label{sec:componentes-del-informe-de-rendimiento-y-capacidad-de-la-plataforma-trii.co}

\begin{quote}
\end{quote}

\begin{figure}
\centering
\includegraphics{images/05.a.Informe.png}
\caption{05.a.Informe. \emph{Fuente: Propuesta de certificación
operativa plataforma Trii.co
(2025)}}\label{fig:id-04abc8f16f354757a52791da825e4049}
\end{figure}

\subsubsection{Información General del Reporte de Rendimiento de
Aplicación
Trii.co}\label{sec:informaciuxf3n-general-del-reporte-de-rendimiento-de-aplicaciuxf3n-trii.co}

Nombre de la Aplicación/Sistema Probado: Servicios de Ordenes, Auth, y
User Info de la Aplicación Trii.co Versión de la Aplicación/Sistema:
Versión 2025 Entorno de Pruebas: infraestructura en la nube, Google
Cloud, 2nd generation machine series, General-purpose workloads E2
serie, CPU Intel. Tipo de equipo: highmem, 7-14 GB. Fecha/Periodo de
Pruebas: 15 de enero del 2025. Objetivos de las Pruebas: * Encontrar la
capacidad de los servicios Servicios Ordenes, Auth, y User Info de la
Aplicación por separado en número máximo de operaciones o transacciones
de los servicios por unidad de tiempo. * Encontrar el estrés o tensión
de los servicios Servicios Ordenes, Auth, y User Info de la Aplicación
por separado para determinar la holgura respecto a la demanda esperada.
* Encontrar el nivel de estabilidad de los servicios Servicios Ordenes,
Auth, y User Info (tensión) de la Aplicación. Métricas Clave: *
Capacidad (throughput) de los servicios Servicios Ordenes, Auth, y User
Info * Estrés (tensión) de los servicios Servicios Ordenes, Auth, y User
Info * Estabilidad (Uso de CPU) de los servicios Servicios Ordenes,
Auth, y User Info Herramienta de Pruebas: K6, de Grafana Labs.

\subsubsection{Datos Específicos del ReporteDescripción detallada de los
casos de uso o flujos de usuario
simulados}\label{sec:datos-especuxedficos-del-reportedescripciuxf3n-detallada-de-los-casos-de-uso-o-flujos-de-usuario-simulados}

\subsubsection{Pruebas de Rendimiento Servicio Get User Info de
Trii.co}\label{sec:pruebas-de-rendimiento-servicio-get-user-info-de-trii.co}

El servicio Get User Info (user info) . Realiza como mínimo actividades
1, 2 y 3 .

Carga de Usuarios: Cantidad de usuarios virtuales concurrentes
simulados, máximo 60. Duración de las Pruebas: Tiempo durante el cual se
ejecutaron las pruebas, mínimo 10 minutos.

Resultados Medidos:

\begin{verbatim}
Escenarios: (100.00%) 1 scenario, 60 max VUs, 10m30s max duration (incl. graceful stop):
*default: Up to 60 looping VUs for 10m0s over 5 stages (gracefulRampDown: 30s, gracefulStop 30s)

logged_in_successfully
is_status_200
\end{verbatim}

\begin{longtable}[]{@{}
  >{\raggedright\arraybackslash}p{(\columnwidth - 12\tabcolsep) * \real{0.1429}}
  >{\raggedright\arraybackslash}p{(\columnwidth - 12\tabcolsep) * \real{0.1429}}
  >{\raggedright\arraybackslash}p{(\columnwidth - 12\tabcolsep) * \real{0.1429}}
  >{\raggedright\arraybackslash}p{(\columnwidth - 12\tabcolsep) * \real{0.1429}}
  >{\raggedright\arraybackslash}p{(\columnwidth - 12\tabcolsep) * \real{0.1429}}
  >{\raggedright\arraybackslash}p{(\columnwidth - 12\tabcolsep) * \real{0.1429}}
  >{\raggedright\arraybackslash}p{(\columnwidth - 12\tabcolsep) * \real{0.1429}}@{}}
\toprule\noalign{}
\endhead
\bottomrule\noalign{}
\endlastfoot
checks & 100.00\% & 57632 out of 57632 & & & & \\
data\_received & 93 MB & 155 kB/s & & & & \\
data\_sent & 14 MB & 23 kB/s & & & & \\
http\_req\_blocked & avg=146.22µs & min=0s & p(95)=0s & p(90)=0s &
max=138.39ms & med=0s \\
http\_req\_connecting & avg=51.45µs & min=0s & p(95)=0s & p(90)=0s &
max=2.39s & med=0s \\
http\_req\_duration & avg=349.97ms & min=184.86ms & p(95)=849.58ms &
p(90)=786.74ms & max=2.39s & med=198.9ms \\
\{ expected\_response:true \} & avg=349.97ms & min=184.86ms &
p(95)=849.58ms & p(90)=786.74ms & max=2.39s & med=198.9ms \\
http\_req\_failed & 0.00\% & 0/57632 & & & & \\
http\_req\_receiving & avg=164.45µs & min=0s & p(95)=1.53ms &
p(90)=546.29µs & max=359.57ms & med=0s \\
http\_req\_sending & avg=66.37µs & min=0s & p(95)=513.59µs & p(90)=0s &
max=2.41ms & med=0s \\
http\_req\_tls\_handshaking & avg=138.25µs & min=0s & p(95)=0s &
p(90)=0s & max=2.39s & med=0s \\
http\_req\_waiting & avg=349.74ms & min=184.86ms & p(95)=849.2ms &
p(90)=786.62ms & max=2.39s & med=198.58ms \\
http\_reqs & 57632 & 960.294772/s & & & & \\
iteration\_duration & avg=708.46ms & min=293.83ms & p(95)=1.15s &
p(90)=987.68ms & max=2.52s & med=657.69ms \\
iterations & 28816 & 479.944772/s & & & & \\
login\_response\_times & avg=136.021551 & min=104 & p(95)=177 &
p(90)=163 & max=296 & med=131 \\
login\_success\_rate & 100.00\% & 28816 out of 28816 & & & & \\
requests\_sent & 57632 & 95.929047/s & & & & \\
user\_info\_response\_times & avg=564.321835 & min=178 & p(95)=1027 &
p(90)=849 & max=2399 & med=519 \\
user\_info\_success\_rate & 100.00\% & 28816 out of 28816 & & & & \\
vus & 59 & min=1 & & & & max=59 \\
vus\_max & 60 & min=60 & & & & max=60 \\
\end{longtable}

\begin{verbatim}
Running (10m00.8s), 00/60 VUs, 28816 completed and 0 interrupted iterations
default OK: 00/60 VUs 10m0s
\end{verbatim}

Valores Numéricos:

\begin{verbatim}
Promedio por transacción, tiempo máximo, mínimo, y percentiles de las métricas. Tomado del mayor entre http_req_duration e iteration_duration.

Tiempo máximo de la transacción (iteración): max=2.52s
Tiempo promedio: avg=708.46ms
Tiempo mínimo: min=293.83ms 
Percentiles 90 y 95 duración iteración: p(90)=987.68ms; p(95)=1.15s
Cantidad de transacciones/segundo (capacidad o throughput): 57632 total; 95.929047/s
Estabilidad o Tasa de éxito de transacción (promedio entre dos servicios, login y user_info): 100.00%; 28816 de 28816 procesados
\end{verbatim}

Desviaciones:

\begin{verbatim}
Comportamiento inesperado o desviaciones significativas de los valores esperados.

Con base en los tiempos de latencia cercanos al tiempo de transacción y la alta la tasa de éxito de transacción, no hay evidencia de desviaciones.

Latencia promedio: avg=349.74ms 
Latencia máxima: max=2.39s
Estabilidad o Tasa de éxito de transacción (promedio entre dos servicios, login y user_info): 100.00%; 28816 de 28816 procesados
\end{verbatim}

Análisis de Cuellos de Botella:

\begin{verbatim}
Identificación de componentes o procesos que limitaron el rendimiento.

Con base en los tiempos de rendimiento (capacidad o throughput) y la alta la tasa de éxito de la transacción, no es posible señalar un cuello de botella.

Cantidad de transacciones/segundo (capacidad o throughput iteración): 57632 total; 95.929047/s
Estabilidad o Tasa de éxito de transacción (promedio entre dos servicios, login y user_info): 100.00%; 28816 de 28816 procesados
\end{verbatim}

Limitaciones:

\begin{verbatim}
Con base en las 28816 iteraciones completadas y 0 interrumpidas, no hubo limitaciones o condiciones conocidas durante las pruebas que podrían afectar los resultados.

Calidad de la prueba: 28816 iteraciones completadas; 0 interrumpidas
\end{verbatim}

\subsubsection{Pruebas de Rendimiento Servicio Auth de
Trii.co}\label{sec:pruebas-de-rendimiento-servicio-auth-de-trii.co}

El servicio Login (auth) es responsable de dar inicio a una sesión de
trabajo de un cliente Trii. Realiza como mínimo actividades 1, 2 y 3 .

Carga de Usuarios: Cantidad de usuarios virtuales concurrentes
simulados, máximo 60. Duración de las Pruebas: Tiempo durante el cual se
ejecutaron las pruebas, mínimo 10 minutos.

Resultados Medidos:

\begin{verbatim}
Escenarios: (100.00%) 1 scenario, 60 max VUs, 10m30s max duration (incl. graceful stop):
*default: Up to 60 looping VUs for 10m0s over 5 stages (gracefulRampDown: 30s, gracefulStop: 30s)

logged_in_successfully
\end{verbatim}

\begin{longtable}[]{@{}
  >{\raggedright\arraybackslash}p{(\columnwidth - 12\tabcolsep) * \real{0.1429}}
  >{\raggedright\arraybackslash}p{(\columnwidth - 12\tabcolsep) * \real{0.1429}}
  >{\raggedright\arraybackslash}p{(\columnwidth - 12\tabcolsep) * \real{0.1429}}
  >{\raggedright\arraybackslash}p{(\columnwidth - 12\tabcolsep) * \real{0.1429}}
  >{\raggedright\arraybackslash}p{(\columnwidth - 12\tabcolsep) * \real{0.1429}}
  >{\raggedright\arraybackslash}p{(\columnwidth - 12\tabcolsep) * \real{0.1429}}
  >{\raggedright\arraybackslash}p{(\columnwidth - 12\tabcolsep) * \real{0.1429}}@{}}
\toprule\noalign{}
\endhead
\bottomrule\noalign{}
\endlastfoot
checks & 100.00\% & 113667 out of 113667 & column\_7 & column\_6 &
column\_5 & column\_4 \\
data\_received & 57 MB & 93 kB/s & & & & \\
data\_sent & 21 MB & 35 kB/s & & & & \\
http\_req\_blocked & avg=37.58µs & min=9s & p(95)=0s & p(90)=0s &
max=92.67ms & med=0s \\
http\_req\_connecting & avg=1.35µs & min=0s & p(95)=0s & p(90)=0s &
max=3.66ms & med=0s \\
http\_req\_duration & avg=177.22ms & min=105.54ms & p(95)=315.41ms &
p(90)=261.69ms & max=3.67s & med=155.66ms \\
\{ expected\_response: true \} & avg=177.22ms & min=105.54ms &
p(95)=315.41ms & p(90)=261.69ms & max=3.67s & med=155.66ms \\
http\_req\_failed & 0.00\% & 0 out of 113677 & & & & \\
http\_req\_receiving & avg=82.42µs & min=9s & p(95)=150.5µs &
p(90)=55.2µs & max=71.71ms & med=0s \\
http\_req\_sending & avg=82.9µs & min=9s & p(95)=569.37µs & p(90)=0s &
max=2.76ms & med=0s \\
http\_req\_tls\_handshaking & avg=36.3µs & min=0s & p(95)=0s & p(90)=0s
& max=87.34ms & med=0s \\
http\_req\_waiting & avg=176.98ms & min=105.54ms & p(95)=315.16ms &
p(90)=261.41ms & max=3.67s & med=155.42ms \\
http\_reqs & 113677 & 189.19272/s & & & & \\
iteration\_duration & avg=177.38ms & min=105.71ms & p(95)=315.52ms &
p(90)=261.84ms & max=3.67s & med=155.8ms \\
iterations & 113677 & 189.19272/s & & & & \\
login\_response\_times & avg=177.3086641 & min=105 & p(95)=316 &
p(90)=262 & max=3675 & med=156 \\
login\_success\_rate & 100.00\% & 113677 out of 113677 & & & & \\
requests\_sent & 113677 & 189.19272/s & & & & \\
vus & 59 & min=1 & & & & max=59 \\
vus\_max & 60 & min=60 & & & & max=60 \\
\end{longtable}

\begin{verbatim}
Running (10m00.2s), 00/60 VUs, 1136777 completed and 0 interrupted iterations
default OK: 00/60 VUs 10m0s
\end{verbatim}

Valores Numéricos:

\begin{verbatim}
Promedio por transacción, tiempo máximo, mínimo, y percentiles de las métricas. Tomado del mayor entre http_req_duration e iteration_duration.

Tiempo máximo de la transacción (iteración): max=3.67s
Tiempo promedio: avg=177.38ms
Tiempo mínimo: min=105.71ms
Percentiles 90 y 95 duración iteración: p(90)=261.84ms; p(95)=315.52ms
Cantidad de transacciones/segundo (capacidad o throughput): 113677 total; 189.19272/s
Estabilidad o Tasa de éxito de transacción: 100.00%; 113677 de 113677 procesados
\end{verbatim}

Nota: el tiempo máximo de transacción, si bien es mayor a 3s, es aún
eficiente debido al tipo de transacción, en este caso de autenticación,
que no compromete al valor del negocio de Trii. Más aún, que el promedio
en este caso no es representativo de la muestra, como sí lo es el valor
del percentil 95: p(95)=315.52ms por transacción.

Desviaciones:

\begin{verbatim}
Comportamiento inesperado o desviaciones significativas de los valores esperados.

Con base en los tiempos de latencia cercanos al tiempo de transacción y la alta la tasa de éxito de transacción, no hay evidencia de desviaciones.

Latencia promedio: avg=176.98ms
Latencia máxima: max=3.67s; p(95)=315.52ms
Estabilidad o Tasa de éxito de transacción: 100.00%; 113677 de 113677 procesados
\end{verbatim}

Nota: el tiempo máximo de latencia, si bien es mayor a 3s, es aún
eficiente debido al tipo de transacción, en este caso de autenticación,
que no compromete al negocio de Trii. Más aún, que el promedio en este
caso no es representativo de la muestra, como sí lo es el valor del
percentil 95: p(95)=315.52ms por transacción.

Análisis de Cuellos de Botella:

\begin{verbatim}
Identificación de componentes o procesos que limitaron el rendimiento.

Latencia máxima: max=3.67s
Latencia promedio: avg=176.98ms
\end{verbatim}

Nota: con base en la diferencia entre la latencia máxima y la promedio
es posible señalar afectación de recursos de la transacción debido a la
concurrencia de los VU (60, para este escenario).

Aún así, por los tiempos de rendimiento (capacidad o throughput) y la
alta la tasa de éxito de la transacción, no es posible señalar que la
posibilidad del cuello de botella en el servicio Auth sea incidente en
el negocio de Trii. Dicho de otra manera, la transacción es resiliente a
pesar de las afectaciones por concurrencia.

\begin{verbatim}
Cantidad de transacciones/segundo (capacidad o throughput): 113677 total; 189.19272/s
Estabilidad o Tasa de éxito de transacción: 100.00%; 113677 de 113677 procesados
\end{verbatim}

Limitaciones:

\begin{verbatim}
No hubo limitaciones o condiciones conocidas durante las pruebas que podrían afectar los resultados.
\end{verbatim}

\subsubsection{Pruebas de Rendimiento Servicio Ordenes de
Trii.co}\label{sec:pruebas-de-rendimiento-servicio-ordenes-de-trii.co}

El servicio Ordenes es el más relevante para el negocio de Trii. Realiza
como mínimo actividades 1, 2 y 3 .

Carga de Usuarios: Cantidad de usuarios virtuales concurrentes
simulados, máximo 60. Duración de las Pruebas: Tiempo durante el cual se
ejecutaron las pruebas, mínimo 10 minutos.

Resultados Medidos:

\begin{verbatim}
Escenarios: (100.00%) 1 scenario, 60 max VUs, 10m30s max duration (incl. graceful stop):
*default: Up to 60 looping VUs for 10m0s over 5 stages (gracefulRampDown: 30s, gracefulStop: 30s)

logged_in_successfully
is_status_200
86%: OK 9881 / ERR 1506
\end{verbatim}

\begin{longtable}[]{@{}
  >{\raggedright\arraybackslash}p{(\columnwidth - 12\tabcolsep) * \real{0.1429}}
  >{\raggedright\arraybackslash}p{(\columnwidth - 12\tabcolsep) * \real{0.1429}}
  >{\raggedright\arraybackslash}p{(\columnwidth - 12\tabcolsep) * \real{0.1429}}
  >{\raggedright\arraybackslash}p{(\columnwidth - 12\tabcolsep) * \real{0.1429}}
  >{\raggedright\arraybackslash}p{(\columnwidth - 12\tabcolsep) * \real{0.1429}}
  >{\raggedright\arraybackslash}p{(\columnwidth - 12\tabcolsep) * \real{0.1429}}
  >{\raggedright\arraybackslash}p{(\columnwidth - 12\tabcolsep) * \real{0.1429}}@{}}
\toprule\noalign{}
\endhead
\bottomrule\noalign{}
\endlastfoot
checks & 93.38\% & 21268 out of 22774 & column\_7 & column\_6 &
column\_5 & column\_4 \\
data\_received & 11 MB & 19 kB/s & & & & \\
data\_sent & 7.5 MB & 13 kB/s & & & & \\
failed\_transactions & avg=1 & min=1 & p(95)=1 & p(90)=1 & max=1 &
med=1 \\
http\_req\_blocked & avg=359.1µs & min=0s & p(95)=8s & p(90)=8s &
max=98.53ms & med=8s \\
http\_req\_connecting & avg=216.56µs & min=0s & p(95)=0s & p(90)=0s &
max=16.6s & med=0s \\
http\_req\_duration & avg=888.53ms & min=197.84ms & p(95)=2.7s &
p(90)=2.21s & max=16.6s & med=309.59ms \\
\{ http\_req\_response:true \} & avg=216.56µs & min=0s & p(95)=2.75s &
p(90)=2.26s & max=16.6s & med=131.43ms \\
http\_req\_failed & 0.61\% & 1506 out of 22774 & & & & \\
http\_req\_receiving & avg=130.7µs & min=0s & p(95)=573.02µs & p(90)=8s
& max=1.99ms & med=8s \\
http\_req\_sending & avg=78.33µs & min=0s & p(95)=573.02µs & p(90)=8s &
max=1.99ms & med=8s \\
http\_req\_tls\_handshaking & avg=284.26µs & min=0s & p(95)=0s &
p(90)=0s & max=16.58s & med=0s \\
http\_req\_waiting & avg=888.32ms & min=197.84ms & p(95)=2.7s &
p(90)=2.21s & max=16.6s & med=309.59ms \\
http\_reqs & 2052/s & 22774 & & & & \\
iteration\_duration & avg=1.77s & min=419.76ms & p(95)=3.31s &
p(90)=2.83s & max=16.74s & med=1.59s \\
iterations & 11387 & 1025.73/s & & & & \\
login\_response\_times & avg=130.558435 & min=197 & p(95)=157 &
p(90)=146 & max=333 & med=128 \\
login\_success\_rate & 188.889.111147 & 2.111147 & & & & \\
requests\_sent & 22774 & 77.71065/s & & & & \\
successful\_transactions & 9881 & 16.36504/s & & & & \\
transaction\_response\_times & avg=1647.048309 & min=388 & p(95)=3178.7
& p(90)=2705.8 & max=16658 & med=1468 \\
vus & 60 & min=60 & & & max=60 & \\
vus\_max & 60 & min=60 & & & max=60 & \\
\end{longtable}

\begin{verbatim}
Running (10m03.8s), 00/60 VUs, 11387 completed and 0 interrupted iterations
default OK: 00/60 VUs 10m0s
\end{verbatim}

Nota: el estado 200 (petición HTML exitosa) en las transacciones del
servicio Ordenes representa además el estado de negocio; es decir, una
transacción correctamente procesada por el sistema, y aceptada por las
reglas de negocio, retorna el estado HTTP 200 en caso que no haya
ocurrido una excepción de negocio. Esto es lo mismo decir que las
transacciones con estado HTML distintas del 200 resultantes en este
escenario de prueba, más precisamente fueron procesadas exitosamente por
el sistema (procesadas sin fallos de sistema evidenciado en logs) aún
cuando hubiesen caído en una regla o excepción de negocio.

Valores Numéricos:

\begin{verbatim}
Promedio por transacción, tiempo máximo, mínimo, y percentiles de las métricas. Tomado del mayor entre http_req_duration e iteration_duration.

Tiempo máximo de la transacción (iteración): max=16.74s; avg p(95/90)=4.49s
Tiempo promedio: avg=1.77s; p(95)=3.31s
Tiempo mínimo: min=419.76ms
Percentiles 90 y 95 duración iteración: p(90)=2.83s; p(95)=3.31s
Cantidad de transacciones/segundo (capacidad o throughput): 22774 total; 16.36504/s
Estabilidad o Tasa de éxito de transacción (iteración): 100.00%; 11387 de 11387. De las cuales 86.00% Ordenes de Negocio (9881 de 11387) exitosas
\end{verbatim}

Nota: el estado 200 (petición HTML exitosa) en las transacciones del
servicio Ordenes representa además el estado de negocio; es decir, una
transacción correctamente procesada por el sistema, y aceptada por las
reglas de negocio, retorna el estado HTTP 200 en caso que no haya
ocurrido una excepción de negocio. Esto es lo mismo decir que las
transacciones con estado HTML distintas del 200 resultantes en este
escenario de prueba, fueron procesadas exitosamente por el sistema
(procesadas sin fallos de sistema evidenciado en logs) aún cuando
hubiesen caído en una regla o excepción de negocio.

Nota: debido a la diferencia entre el tiempo promedio de la transacción
y el percentil 95, el tiempo máximo de transacción no es representativo
de la muestra. Tomaremos como valor máximo de la transacción de Ordenes
al promedio de los percentiles 90 y 95, que es p(95/90)=4.49s.

Nota: el tiempo máximo de transacción (iteración) de Ordenes, si bien es
mayor a 3s, es aún eficiente debido a la complejidad de la transacción y
que no está comprometiendo al negocio de Trii evidenciado en la
estabilidad del 100\% de este servicio y en el percentil 95 de duración,
que sí es representativo, y es de p(95)=3.31s por transacción.

Desviaciones:

\begin{verbatim}
Comportamiento inesperado o desviaciones significativas de los valores esperados.

Con base en los tiempos de latencia cercanos al tiempo de transacción y la alta la tasa de éxito de transacción, no hay evidencia de desviaciones.

Latencia promedio: avg=888.32ms; p(95)=2.7s
Latencia máxima: max=16.6s; avg p(95/90)=3.8s
Estabilidad o Tasa de éxito de transacción (iteración): 100.00%; 11387 de 11387. De las cuales 86.00% Ordenes de Negocio (9881 de 11387) exitosas
\end{verbatim}

Nota: Debido a la diferencia entre la latencia promedio y su percentil
95, el tiempo máximo de latencia por transacción no es representativo de
la muestra. Tomaremos como valor máximo de la latencia de Ordenes al
promedio de los percentiles 90 y 95, que es p(95/90)=3.8s.

Nota: el tiempo máximo de latencia, si bien es mayor a 3s, es aún
eficiente debido a la complejidad de la transacción Ordenes y no está
comprometiendo al negocio de Trii evidenciado en la estabilidad del
100\% de este servicio y en el percentil 95 de latencia, que sí es
representativo, y es de p(95)=2.7s por transacción.

Análisis de Cuellos de Botella:

\begin{verbatim}
Identificación de componentes o procesos que limitaron el rendimiento.

Latencia máxima: max=16.6s; avg p(95/90)=3.8s
Latencia promedio: avg=888.32ms; p(95)=2.7s
\end{verbatim}

Nota: con base en la diferencia entre la latencia máxima y la promedio
existe alta posibilidad de recursos de la transacción Ordenes afectados
por la concurrencia de los VU (60, para este escenario).

Aún así, los tiempos de rendimiento (capacidad o throughput),
16.36504/s, y la tasa de éxito de la transacción, no es posible señalar
que la posibilidad del cuello de botella esté afectando al negocio de
Trii. Dicho de otra manera, la transacción Ordenes es resiliente a pesar
de las afectaciones por concurrencia.

\begin{verbatim}
Cantidad de transacciones/segundo (capacidad o throughput): 22774 total; 16.36504/s
Estabilidad o Tasa de éxito de transacción (iteración): 100.00%; 11387 de 11387. De las cuales 86.00% Ordenes de Negocio (9881 de 11387) exitosas
\end{verbatim}

Limitaciones:

\begin{verbatim}
En este escenario existieron limitaciones o condiciones conocidas del balance de Ordenes durante las pruebas que afectaron los resultados de las métricas de transacción exitosa.

is_status_200
86%: OK 9881 / ERR 1506
Estabilidad o Tasa de éxito de transacción (iteración): 100.00%; 11387 de 11387. De las cuales 86.00% Ordenes de Negocio (9881 de 11387) exitosas    
\end{verbatim}

El estado 200 (petición HTML exitosa) en las transacciones del servicio
Ordenes representa además el estado de negocio; es decir, una
transacción correctamente procesada por el sistema, y aceptada por las
reglas de negocio, retorna el estado HTTP 200 en caso que no haya
ocurrido una excepción de negocio. Esto es lo mismo decir que las
transacciones con estado HTML distintas del 200 resultantes en este
escenario de prueba, fueron procesadas exitosamente por el sistema
(procesadas sin fallos de sistema evidenciado en logs) aún cuando
hubiesen caído en una regla o excepción de negocio.

\subsubsection{Referencias}\label{sec:referencias}

\begin{enumerate}
\def\labelenumi{\arabic{enumi}.}
\tightlist
\item
  https://cloud.google.com/compute/docs/machine-resource
\item
  https://grafana.com/docs/k6/latest/
\item
  https://www.techtarget.com/searchsoftwarequality/tip/Acceptable-application-response-times-vs-industry-standard
\item
  https://www.nngroup.com/articles/response-times-3-important-limits/
\end{enumerate}

\newpage

\section{Evaluación del
Rendimiento}\label{sec:evaluaciuxf3n-del-rendimiento}

\subsection{Método de Evaluación del Rendimiento Actual de
Trii.co}\label{sec:muxe9todo-de-evaluaciuxf3n-del-rendimiento-actual-de-trii.co}

\begin{quote}
\end{quote}

\subsubsection{Criterios de Evaluación del Rendimiento
Actual}\label{sec:criterios-de-evaluaciuxf3n-del-rendimiento-actual}

\begin{longtable}[]{@{}
  >{\raggedright\arraybackslash}p{(\columnwidth - 8\tabcolsep) * \real{0.0930}}
  >{\raggedright\arraybackslash}p{(\columnwidth - 8\tabcolsep) * \real{0.0465}}
  >{\raggedright\arraybackslash}p{(\columnwidth - 8\tabcolsep) * \real{0.1070}}
  >{\raggedright\arraybackslash}p{(\columnwidth - 8\tabcolsep) * \real{0.6977}}
  >{\raggedright\arraybackslash}p{(\columnwidth - 8\tabcolsep) * \real{0.0558}}@{}}
\toprule\noalign{}
\begin{minipage}[b]{\linewidth}\raggedright
Prueba
\end{minipage} & \begin{minipage}[b]{\linewidth}\raggedright
Servicio
\end{minipage} & \begin{minipage}[b]{\linewidth}\raggedright
Modalidad
\end{minipage} & \begin{minipage}[b]{\linewidth}\raggedright
Criterio de Aceptación
\end{minipage} & \begin{minipage}[b]{\linewidth}\raggedright
Ambiente
\end{minipage} \\
\midrule\noalign{}
\endhead
\bottomrule\noalign{}
\endlastfoot
Login & Auth & Concurrente, Unitaria & Percentil de peticiones exitosas
99.9. Tiempo de respuesta max 4 seg. Tasa procesamiento (throughput):
Transacciones por hora 2500 y 40 por minuto & Dev \\
Get user info & Auth & Concurrente, Integral & Percentil de peticiones
exitosas 99.9. Tiempo de respuesta max 4 seg. Tasa procesamiento
(throughput): Transacciones por hora 2500 y 40 por minuto & Dev \\
Fee & Auth & Concurrente, Integral & Percentil de peticiones exitosas
99.9. Tiempo de respuesta max 4 seg. Tasa procesamiento:(throughput):
Transacciones por hora 2500 y 40 por minuto & Dev \\
Ingreso de órdenes & Órdenes & Concurrente, Integral & Percentil de
peticiones exitosas 99.9. Tiempo de respuesta max 4.5 seg. Tasa
procesamiento:(throughput): Transacciones por hora 2500 y 40 por minuto
& Dev / Prod \\
\end{longtable}

En donde: * Transacciones diarias: 10000/d * Transacciones por hora
(10000 / 4): 2500/h * Transacciones por minuto: 40/m * Transacciones por
segundo: 4/s

\newpage

\section{Resultados y Conclusiones del Informe de
Rendimiento}\label{sec:resultados-y-conclusiones-del-informe-de-rendimiento}

\subsection{Análisis de Resultados del Rendimiento y
Capacidad}\label{sec:anuxe1lisis-de-resultados-del-rendimiento-y-capacidad}

\begin{quote}
\end{quote}

\subsubsection{Compilación de Resultado de las Pruebas de
Rendimiento}\label{sec:compilaciuxf3n-de-resultado-de-las-pruebas-de-rendimiento}

\begin{longtable}[]{@{}
  >{\raggedright\arraybackslash}p{(\columnwidth - 4\tabcolsep) * \real{0.1053}}
  >{\raggedright\arraybackslash}p{(\columnwidth - 4\tabcolsep) * \real{0.4105}}
  >{\raggedright\arraybackslash}p{(\columnwidth - 4\tabcolsep) * \real{0.4842}}@{}}
\toprule\noalign{}
\begin{minipage}[b]{\linewidth}\raggedright
Prueba
\end{minipage} & \begin{minipage}[b]{\linewidth}\raggedright
Criterio de Aceptación
\end{minipage} & \begin{minipage}[b]{\linewidth}\raggedright
Resultado
\end{minipage} \\
\midrule\noalign{}
\endhead
\bottomrule\noalign{}
\endlastfoot
Login & Percentil de peticiones exitosas 99.9 & Estabilidad o Tasa de
éxito de transacción: 100.00\%; 113677 de 113677 procesados \\
Login & Tiempo de respuesta max 4 seg. & Tiempo máximo de la transacción
(iteración): max=3.67s \\
Login & Tasa procesamiento (throughput), 2500 transacciones por hora y
40 por minuto & Cantidad de transacciones/segundo (capacidad o
throughput): 113677 total; 189.19272/s \\
Get user info & Percentil de peticiones exitosas 99.9 & Estabilidad o
Tasa de éxito de transacción: 100.00\%; 28816 de 28816 procesados \\
Get user info & Tiempo de respuesta max 4 seg. & Tiempo máximo de la
transacción (iteración): max=2.52s \\
Get user info & Tasa procesamiento (throughput): 2500 transacciones por
hora y 40 por minuto & Cantidad de transacciones/segundo (capacidad o
throughput): 57632 total; 95.929047/s \\
Fee & Percentil de peticiones exitosas 99.9 & Estabilidad o Tasa de
éxito de transacción: 100.00\%; 28816 de 28816 procesados \\
Fee & Tiempo de respuesta max 4 seg. & Tiempo máximo de la transacción
(iteración): max=2.52s \\
Fee & Tasa procesamiento (throughput): 2500 transacciones por hora y 40
por minuto & Cantidad de transacciones/segundo (capacidad o throughput):
57632 total; 95.929047/s \\
Ingreso de órdenes & Percentil de peticiones exitosas 99.9 & Estabilidad
o Tasa de éxito de transacción (iteración): 100.00\%; 11387 de 11387
procesados \\
Ingreso de órdenes & Tiempo de respuesta max 4.5 seg. & Tiempo máximo de
la transacción (iteración): max=16.74s; avg p(95/90)=4.49s \\
Ingreso de órdenes & Tasa procesamiento (throughput): 2500 transacciones
por hora y 40 por minuto & Cantidad de transacciones/segundo (capacidad
o throughput): 22774 total; 16.36504/s \\
\end{longtable}

El resultado de las pruebas de rendimiento ejecutadas para los servicios
de la Aplicación Trii.co, Login, Get User Info, Fee, Ordenes, comprueba
que la capacidad operativa, en términos de rendimientos, estabilidad y
respuesta, está por encima de lo generalmente aceptados por los
estándares de tiempo de respuesta de aplicaciones de software
empresariales, en este caso particular, de tipo web para la industria
fintech.

\begin{verbatim}
10 seconds is about the limit for keeping the user's attention focused on the dialogue. For longer delays, users will want to perform other tasks while waiting for the computer to finish, so they should be given feedback indicating when the computer expects to be done. Feedback during the delay is especially important if the response time is likely to be highly variable, since users will then not know what to expect. -- Nielsen, J. (1993). Usability Engineering. Response Times: The 3 Important Limits (web).
\end{verbatim}

\subsubsection{Conclusión General}\label{sec:conclusiuxf3n-general}

Teniendo de base los resultados de la actual prueba de rendimiento
consignados en este informe, es factible indicar que el umbral de
crecimiento de Trii, sin que alcance a comprometer la estabilidad de la
Aplicación, en términos de nivel de ocupación de recursos y tasa de
éxito, podría llegar a ser de entre el 4x y 5x de la carga de
procesamiento real actual. Es decir, con la capacidad operativa actual,
sin requerir inversión en su plan de capacidad, podría aumentar sus
niveles de procesamiento en un 400\% (esto es, de \textasciitilde5000
transacciones diarias a 22774), como mínimo, sin comprometer la
estabilidad del sistema completo.

\newpage

\section{Anexos Técnicos}\label{sec:anexos-tuxe9cnicos}

\begin{enumerate}
\def\labelenumi{\arabic{enumi}.}
\tightlist
\item
  Archivos de registro de actividad
\item
  Evidencia de la ocupación de recursos
\item
  Referencias
\end{enumerate}

\end{document}
