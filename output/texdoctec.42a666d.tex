%%
% Copyright (c) 2017 - 2024, Pascal Wagler;
% Copyright (c) 2014 - 2024, John MacFarlane
%
% All rights reserved.
%
% Redistribution and use in source and binary forms, with or without
% modification, are permitted provided that the following conditions
% are met:
%
% - Redistributions of source code must retain the above copyright
% notice, this list of conditions and the following disclaimer.
%
% - Redistributions in binary form must reproduce the above copyright
% notice, this list of conditions and the following disclaimer in the
% documentation and/or other materials provided with the distribution.
%
% - Neither the name of John MacFarlane nor the names of other
% contributors may be used to endorse or promote products derived
% from this software without specific prior written permission.
%
% THIS SOFTWARE IS PROVIDED BY THE COPYRIGHT HOLDERS AND CONTRIBUTORS
% "AS IS" AND ANY EXPRESS OR IMPLIED WARRANTIES, INCLUDING, BUT NOT
% LIMITED TO, THE IMPLIED WARRANTIES OF MERCHANTABILITY AND FITNESS
% FOR A PARTICULAR PURPOSE ARE DISCLAIMED. IN NO EVENT SHALL THE
% COPYRIGHT OWNER OR CONTRIBUTORS BE LIABLE FOR ANY DIRECT, INDIRECT,
% INCIDENTAL, SPECIAL, EXEMPLARY, OR CONSEQUENTIAL DAMAGES (INCLUDING,
% BUT NOT LIMITED TO, PROCUREMENT OF SUBSTITUTE GOODS OR SERVICES;
% LOSS OF USE, DATA, OR PROFITS; OR BUSINESS INTERRUPTION) HOWEVER
% CAUSED AND ON ANY THEORY OF LIABILITY, WHETHER IN CONTRACT, STRICT
% LIABILITY, OR TORT (INCLUDING NEGLIGENCE OR OTHERWISE) ARISING IN
% ANY WAY OUT OF THE USE OF THIS SOFTWARE, EVEN IF ADVISED OF THE
% POSSIBILITY OF SUCH DAMAGE.
%%

%%
% This is the Eisvogel pandoc LaTeX template.
%
% For usage information and examples visit the official GitHub page:
% https://github.com/Wandmalfarbe/pandoc-latex-template
%%

% Options for packages loaded elsewhere
\PassOptionsToPackage{unicode}{hyperref}
\PassOptionsToPackage{hyphens}{url}
\PassOptionsToPackage{dvipsnames,svgnames,x11names,table}{xcolor}
%
\documentclass[
  paper=a4,
  ,captions=tableheading
]{scrartcl}
\usepackage{amsmath,amssymb}
% Use setspace anyway because we change the default line spacing.
% The spacing is changed early to affect the titlepage and the TOC.
\usepackage{setspace}
\setstretch{1.2}
\usepackage{iftex}
\ifPDFTeX
  \usepackage[T1]{fontenc}
  \usepackage[utf8]{inputenc}
  \usepackage{textcomp} % provide euro and other symbols
\else % if luatex or xetex
  \usepackage{unicode-math} % this also loads fontspec
  \defaultfontfeatures{Scale=MatchLowercase}
  \defaultfontfeatures[\rmfamily]{Ligatures=TeX,Scale=1}
\fi
\usepackage{lmodern}
\ifPDFTeX\else
  % xetex/luatex font selection
\fi
% Use upquote if available, for straight quotes in verbatim environments
\IfFileExists{upquote.sty}{\usepackage{upquote}}{}
\IfFileExists{microtype.sty}{% use microtype if available
  \usepackage[]{microtype}
  \UseMicrotypeSet[protrusion]{basicmath} % disable protrusion for tt fonts
}{}
\makeatletter
\@ifundefined{KOMAClassName}{% if non-KOMA class
  \IfFileExists{parskip.sty}{%
    \usepackage{parskip}
  }{% else
    \setlength{\parindent}{0pt}
    \setlength{\parskip}{6pt plus 2pt minus 1pt}}
}{% if KOMA class
  \KOMAoptions{parskip=half}}
\makeatother
\usepackage{xcolor}
\definecolor{default-linkcolor}{HTML}{A50000}
\definecolor{default-filecolor}{HTML}{A50000}
\definecolor{default-citecolor}{HTML}{4077C0}
\definecolor{default-urlcolor}{HTML}{4077C0}
\usepackage[top=1.3in,bottom=1in,left=0.7in,right=0.7in]{geometry}
\usepackage[export]{adjustbox}
\usepackage{graphicx}
\usepackage{longtable,booktabs,array}
\usepackage{calc} % for calculating minipage widths
% Correct order of tables after \paragraph or \subparagraph
\usepackage{etoolbox}
\makeatletter
\patchcmd\longtable{\par}{\if@noskipsec\mbox{}\fi\par}{}{}
\makeatother
% Allow footnotes in longtable head/foot
\IfFileExists{footnotehyper.sty}{\usepackage{footnotehyper}}{\usepackage{footnote}}
\makesavenoteenv{longtable}
% add backlinks to footnote references, cf. https://tex.stackexchange.com/questions/302266/make-footnote-clickable-both-ways
\usepackage{footnotebackref}
\setlength{\emergencystretch}{3em} % prevent overfull lines
\providecommand{\tightlist}{%
  \setlength{\itemsep}{0pt}\setlength{\parskip}{0pt}}
\setcounter{secnumdepth}{-\maxdimen} % remove section numbering
\ifLuaTeX
\usepackage[bidi=basic]{babel}
\else
\usepackage[bidi=default]{babel}
\fi
\babelprovide[main,import]{english}
% get rid of language-specific shorthands (see #6817):
\let\LanguageShortHands\languageshorthands
\def\languageshorthands#1{}
\makeatletter
\@ifpackageloaded{subfig}{}{\usepackage{subfig}}
\@ifpackageloaded{caption}{}{\usepackage{caption}}
\captionsetup[subfloat]{margin=0.5em}
\AtBeginDocument{%
\renewcommand*\figurename{Figura}
\renewcommand*\tablename{Tabla}
}
\AtBeginDocument{%
\renewcommand*\listfigurename{Lista de Figuras}
\renewcommand*\listtablename{Lista de Tablas}
}
\newcounter{pandoccrossref@subfigures@footnote@counter}
\newenvironment{pandoccrossrefsubfigures}{%
\setcounter{pandoccrossref@subfigures@footnote@counter}{0}
\begin{figure}\centering%
\gdef\global@pandoccrossref@subfigures@footnotes{}%
\DeclareRobustCommand{\footnote}[1]{\footnotemark%
\stepcounter{pandoccrossref@subfigures@footnote@counter}%
\ifx\global@pandoccrossref@subfigures@footnotes\empty%
\gdef\global@pandoccrossref@subfigures@footnotes{{##1}}%
\else%
\g@addto@macro\global@pandoccrossref@subfigures@footnotes{, {##1}}%
\fi}}%
{\end{figure}%
\addtocounter{footnote}{-\value{pandoccrossref@subfigures@footnote@counter}}
\@for\f:=\global@pandoccrossref@subfigures@footnotes\do{\stepcounter{footnote}\footnotetext{\f}}%
\gdef\global@pandoccrossref@subfigures@footnotes{}}
\@ifpackageloaded{float}{}{\usepackage{float}}
\floatstyle{ruled}
\@ifundefined{c@chapter}{\newfloat{codelisting}{h}{lop}}{\newfloat{codelisting}{h}{lop}[chapter]}
\floatname{codelisting}{Listing}
\newcommand*\listoflistings{\listof{codelisting}{Listas del Documento}}
\makeatother
\usepackage{bookmark}
\IfFileExists{xurl.sty}{\usepackage{xurl}}{} % add URL line breaks if available
\urlstyle{same}
\hypersetup{
  pdftitle={Certificación Operativa Plataforma de Software Trii.co},
  pdfauthor={SoftProductiva.com},
  pdflang={en},
  pdfsubject={Implementación Proyecto},
  pdfkeywords={Rendimiento, Métodos pruebas, Pruebas software, QA},
  hidelinks,
  breaklinks=true,
  pdfcreator={LaTeX via pandoc with the Eisvogel template}}
\title{Certificación Operativa Plataforma de Software Trii.co}
\usepackage{etoolbox}
\makeatletter
\providecommand{\subtitle}[1]{% add subtitle to \maketitle
  \apptocmd{\@title}{\par {\large #1 \par}}{}{}
}
\makeatother
\subtitle{Informe Ejecutivo}
\author{SoftProductiva.com}
\date{2025-01-20}



%%
%% added
%%


%
% for the background color of the title page
%
\usepackage{pagecolor}
\usepackage{afterpage}
\usepackage{tikz}

%
% break urls
%
\PassOptionsToPackage{hyphens}{url}

%
% When using babel or polyglossia with biblatex, loading csquotes is recommended
% to ensure that quoted texts are typeset according to the rules of your main language.
%
\usepackage{csquotes}

%
% captions
%
\definecolor{caption-color}{HTML}{777777}
\usepackage[font={stretch=1.2}, textfont={color=caption-color}, position=top, skip=4mm, labelfont=bf, singlelinecheck=false, justification=raggedright]{caption}
\setcapindent{0em}

%
% blockquote
%
\definecolor{blockquote-border}{RGB}{221,221,221}
\definecolor{blockquote-text}{RGB}{119,119,119}
\usepackage{mdframed}
\newmdenv[rightline=false,bottomline=false,topline=false,linewidth=3pt,linecolor=blockquote-border,skipabove=\parskip]{customblockquote}
\renewenvironment{quote}{\begin{customblockquote}\list{}{\rightmargin=0em\leftmargin=0em}%
\item\relax\color{blockquote-text}\ignorespaces}{\unskip\unskip\endlist\end{customblockquote}}

%
% Source Sans Pro as the default font family
% Source Code Pro for monospace text
%
% 'default' option sets the default
% font family to Source Sans Pro, not \sfdefault.
%
\ifnum 0\ifxetex 1\fi\ifluatex 1\fi=0 % if pdftex
    \usepackage[default]{sourcesanspro}
  \usepackage{sourcecodepro}
  \else % if not pdftex
    \usepackage[default]{sourcesanspro}
  \usepackage{sourcecodepro}

  % XeLaTeX specific adjustments for straight quotes: https://tex.stackexchange.com/a/354887
  % This issue is already fixed (see https://github.com/silkeh/latex-sourcecodepro/pull/5) but the
  % fix is still unreleased.
  % TODO: Remove this workaround when the new version of sourcecodepro is released on CTAN.
  \ifxetex
    \makeatletter
    \defaultfontfeatures[\ttfamily]
      { Numbers   = \sourcecodepro@figurestyle,
        Scale     = \SourceCodePro@scale,
        Extension = .otf }
    \setmonofont
      [ UprightFont    = *-\sourcecodepro@regstyle,
        ItalicFont     = *-\sourcecodepro@regstyle It,
        BoldFont       = *-\sourcecodepro@boldstyle,
        BoldItalicFont = *-\sourcecodepro@boldstyle It ]
      {SourceCodePro}
    \makeatother
  \fi
  \fi

%
% heading color
%
\definecolor{heading-color}{RGB}{40,40,40}
\addtokomafont{section}{\color{heading-color}}
% When using the classes report, scrreprt, book,
% scrbook or memoir, uncomment the following line.
%\addtokomafont{chapter}{\color{heading-color}}

%
% variables for title, author and date
%
\usepackage{titling}
\title{Certificación Operativa Plataforma de Software Trii.co}
\author{SoftProductiva.com}
\date{2025-01-20}

%
% tables
%

\definecolor{table-row-color}{HTML}{F5F5F5}
\definecolor{table-rule-color}{HTML}{999999}

%\arrayrulecolor{black!40}
\arrayrulecolor{table-rule-color}     % color of \toprule, \midrule, \bottomrule
\setlength\heavyrulewidth{0.3ex}      % thickness of \toprule, \bottomrule
\renewcommand{\arraystretch}{1.3}     % spacing (padding)


%
% remove paragraph indentation
%
\setlength{\parindent}{0pt}
\setlength{\parskip}{6pt plus 2pt minus 1pt}
\setlength{\emergencystretch}{3em}  % prevent overfull lines

%
%
% Listings
%
%


%
% header and footer
%
\usepackage[headsepline,footsepline]{scrlayer-scrpage}

\newpairofpagestyles{eisvogel-header-footer}{
  \clearpairofpagestyles
  \ihead*{2025-01-20}
  \chead*{}
  \ohead*{Certificación Operativa Plataforma de Software Trii.co}
  \ifoot*{SoftProductiva.com}
  \cfoot*{}
  \ofoot*{\thepage}
  \addtokomafont{pageheadfoot}{\upshape}
}
\pagestyle{eisvogel-header-footer}



%%
%% end added
%%

%% change the default with of the images using \setkeys{Gin}{width=.8\linewidth}
\setkeys{Gin}{width=1\linewidth}

\begin{document}

%%
%% begin titlepage
%%
\begin{titlepage}
\newgeometry{top=2cm, right=4cm, bottom=3cm, left=4cm}
\tikz[remember picture,overlay] \node[inner sep=0pt] at (current page.center){\includegraphics[width=\paperwidth,height=\paperheight]{include/background.pdf}};
\newcommand{\colorRule}[3][black]{\textcolor[HTML]{#1}{\rule{#2}{#3}}}
\begin{flushleft}
\noindent
\\[-1em]
\color[HTML]{5F5F5F}
\makebox[0pt][l]{\colorRule[360049]{1.3\textwidth}{4pt}}
\par
\noindent

% The titlepage with a background image has other text spacing and text size
{
  \setstretch{2}
  \vfill
  \vskip -8em
  \noindent {\huge \textbf{\textsf{Certificación Operativa Plataforma de
Software Trii.co}}}
    \vskip 1em
  {\Large \textsf{Informe Ejecutivo}}
    \vskip 2em
  \noindent {\Large \textsf{SoftProductiva.com} \vskip 0.6em \textsf{2025-01-20}}
  \vfill
}

\noindent
\includegraphics[width=60mm, right]{include/logo.png}

\end{flushleft}
\end{titlepage}
\restoregeometry
\pagenumbering{arabic}

%%
%% end titlepage
%%

% \maketitle


\section{Contenido}\label{sec:contenido}

\begin{itemize}
\tightlist
\item
  \hyperref[informaciuxf3n-del-documento]{Información del Documento}
\item
  \hyperref[resumen-ejecutivo]{Resumen Ejecutivo}
\item
  \hyperref[informe-de-rendimiento-y-capacidad]{Informe de Rendimiento y
  Capacidad}
\item
  \hyperref[resultados-y-conclusiones-del-informe-de-rendimiento]{Resultados
  y Conclusiones del Informe de Rendimiento}
\end{itemize}

\newpage

\section{Información del
Documento}\label{sec:informaciuxf3n-del-documento}

\subsection{Versión del Documento}\label{sec:versiuxf3n-del-documento}

\begin{quote}
\end{quote}

\subsection{Control de Cambios}\label{sec:control-de-cambios}

Historia de cambios del informe.

Versión actual: 1.42a666d - Compilación para entrega - Tue, 21 Jan 2025
23:00:48 +0000

Versiones Anteriores

1.8087c3f - Compilación para entrega - Tue, 21 Jan 2025 19:54:22 +0000

1.938399b - Compilación para entrega - Tue, 21 Jan 2025 18:24:45 +0000

1.8a9bfb2 - Compilación para entrega - Mon, 20 Jan 2025 22:26:14 +0000

1.d3a0900 - Compilación para entrega - Mon, 20 Jan 2025 22:07:25 +0000

\subsubsection{Realizado Por}\label{sec:realizado-por}

H. Wong, ing.

\subsubsection{Revisado Por}\label{sec:revisado-por}

(revisor), Trii.co

\newpage

\section{Resumen Ejecutivo}\label{sec:resumen-ejecutivo}

\subsection{}\label{sec:section}

\begin{quote}
\end{quote}

\subsubsection{Resumen del Informe de Rendimiento
Trii.co}\label{sec:resumen-del-informe-de-rendimiento-trii.co}

El Informe Técnico SoftProductiva.com certifica el rendimiento operativo
de la plataforma de software Trii.co. Se evaluaron los servicios
relevantes de la plataforma, inicio de sesión ``Login,'', requisición de
información de trabajo, ``Get User Info,'' y el de registro de órden
``Órdenes'', mediante técnicas de pruebas de rendimiento con la
herramienta K6 de Grafala Labs. Los resultados muestran una alta
estabilidad y capacidad de procesamiento que supera los criterios de
aceptación preestablecidos, con un margen de crecimiento significativo
antes de alcanzar limitaciones operativas y sin requerir inversión en el
plan de capacidad. El informe concluye que Trii.co tiene una holgura
considerable en su capacidad actual.

\newpage

\section{Informe de Rendimiento y
Capacidad}\label{sec:informe-de-rendimiento-y-capacidad}

\subsection{Componentes del Informe de Rendimiento y Capacidad de la
Plataforma
Trii.co}\label{sec:componentes-del-informe-de-rendimiento-y-capacidad-de-la-plataforma-trii.co}

\begin{quote}
\end{quote}

\subsubsection{Información General del Reporte de Rendimiento de
Aplicación
Trii.co}\label{sec:informaciuxf3n-general-del-reporte-de-rendimiento-de-aplicaciuxf3n-trii.co}

\begin{itemize}
\tightlist
\item
  Nombre de la Aplicación/Sistema Probado: Servicios de Ordenes, Auth, y
  User Info de la Aplicación Trii.co
\item
  Versión de la Aplicación/Sistema: Versión 2025
\item
  Entorno de Pruebas: infraestructura en la nube, Google Cloud, 2nd
  generation machine series, General-purpose workloads E2 serie, CPU
  Intel. Tipo de equipo: highmem, 7-14 GB.
\item
  Fecha/Periodo de Pruebas: 15 de enero del 2025.
\item
  Objetivos de las Pruebas:

  \begin{itemize}
  \tightlist
  \item
    Encontrar la capacidad de los servicios Servicios Ordenes, Auth, y
    User Info de la Aplicación por separado en número máximo de
    operaciones o transacciones de los servicios por unidad de tiempo.
  \item
    Encontrar el nivel de estabilidad de los servicios Servicios
    Ordenes, Auth, y User Info (tensión) de la Aplicación.
  \item
    Dar pautas alrededor del estrés o tensión de los servicios Servicios
    Ordenes, Auth, y User Info de la Aplicación por separado para
    determinar la holgura respecto a la demanda esperada.
  \end{itemize}
\item
  Métricas Clave:

  \begin{itemize}
  \tightlist
  \item
    Capacidad (throughput) de los servicios Servicios Ordenes, Auth, y
    User Info
  \item
    Estrés (tensión) de los servicios Servicios Ordenes, Auth, y User
    Info
  \item
    Estabilidad (Uso de CPU) de los servicios Servicios Ordenes, Auth, y
    User Info Herramienta de Pruebas: K6, de Grafana Labs.
  \end{itemize}
\end{itemize}

\newpage

\section{Resultados y Conclusiones del Informe de
Rendimiento}\label{sec:resultados-y-conclusiones-del-informe-de-rendimiento}

\subsection{Análisis de Resultados del Rendimiento y
Capacidad}\label{sec:anuxe1lisis-de-resultados-del-rendimiento-y-capacidad}

\begin{quote}
\end{quote}

\subsubsection{Resumen y Puntos Sobresalientes de los
Resultados}\label{sec:resumen-y-puntos-sobresalientes-de-los-resultados}

\begin{enumerate}
\def\labelenumi{\arabic{enumi}.}
\tightlist
\item
  Todos los servicios probados (auth, user\_info, fee y ordenes) pasaron
  los criterios de aceptación de estabilidad, tiempo de respuesta, y
  capacidad de cómputo (throughput). Pag. 14, Informe Técnico
\item
  El análisis de latencia del servicio de Ordenes indica una alta
  posibilidad de que exista un cuello botella, pero no afecta la
  estabilidad del servicio: cero (0) fallas en registro de actividad del
  sistema. Pág. 11, Informe Técnico; razón por la cual
\item
  El servicio de órdenes requirió del ajuste en el criterio de
  aceptación \emph{tiempo de respuesta}: quedó en 4.5s. Pág. 10, Informe
  Técnico
\item
  La conclusión general del rendimiento de Trii.co actual, `como está',
  sin inversión de capacidad, presenta holgura del 4x. Es decir, sin
  cambios en el plan de capacidad Trii puede crecer un 400\% del
  rendimiento actual. Pág. 15, Informe Técnico
\end{enumerate}

\subsubsection{Compilación de Resultado de las Pruebas de
Rendimiento}\label{sec:compilaciuxf3n-de-resultado-de-las-pruebas-de-rendimiento}

\begin{longtable}[]{@{}
  >{\raggedright\arraybackslash}p{(\columnwidth - 4\tabcolsep) * \real{0.1053}}
  >{\raggedright\arraybackslash}p{(\columnwidth - 4\tabcolsep) * \real{0.4105}}
  >{\raggedright\arraybackslash}p{(\columnwidth - 4\tabcolsep) * \real{0.4842}}@{}}
\toprule\noalign{}
\begin{minipage}[b]{\linewidth}\raggedright
Prueba
\end{minipage} & \begin{minipage}[b]{\linewidth}\raggedright
Criterio de Aceptación
\end{minipage} & \begin{minipage}[b]{\linewidth}\raggedright
Resultado
\end{minipage} \\
\midrule\noalign{}
\endhead
\bottomrule\noalign{}
\endlastfoot
Login & Percentil de peticiones exitosas 99.9 & Estabilidad o Tasa de
éxito de transacción: 100.00\%; 113677 de 113677 procesados \\
Login & Tiempo de respuesta max 4 seg. & Tiempo máximo de la transacción
(iteración): max=3.67s \\
Login & Tasa procesamiento (throughput), 2500 transacciones por hora y
40 por minuto & Cantidad de transacciones/segundo (capacidad o
throughput): 113677 total; 189.19272/s \\
Get user info & Percentil de peticiones exitosas 99.9 & Estabilidad o
Tasa de éxito de transacción: 100.00\%; 28816 de 28816 procesados \\
Get user info & Tiempo de respuesta max 4 seg. & Tiempo máximo de la
transacción (iteración): max=2.52s \\
Get user info & Tasa procesamiento (throughput): 2500 transacciones por
hora y 40 por minuto & Cantidad de transacciones/segundo (capacidad o
throughput): 57632 total; 95.929047/s \\
Fee & Percentil de peticiones exitosas 99.9 & Estabilidad o Tasa de
éxito de transacción: 100.00\%; 28816 de 28816 procesados \\
Fee & Tiempo de respuesta max 4 seg. & Tiempo máximo de la transacción
(iteración): max=2.52s \\
Fee & Tasa procesamiento (throughput): 2500 transacciones por hora y 40
por minuto & Cantidad de transacciones/segundo (capacidad o throughput):
57632 total; 95.929047/s \\
Ingreso de órdenes & Percentil de peticiones exitosas 99.9 & Estabilidad
o Tasa de éxito de transacción (iteración): 100.00\%; 11387 de 11387
procesados \\
Ingreso de órdenes & Tiempo de respuesta max 4.5 seg. & Tiempo máximo de
la transacción (iteración): max=16.74s; avg p(95/90)=4.49s \\
Ingreso de órdenes & Tasa procesamiento (throughput): 2500 transacciones
por hora y 40 por minuto & Cantidad de transacciones/segundo (capacidad
o throughput): 22774 total; 16.36504/s \\
\end{longtable}

El resultado de las pruebas de rendimiento ejecutadas para los servicios
de la Aplicación Trii.co, Login, Get User Info, Fee, Ordenes, comprueba
que la capacidad operativa, en términos de rendimientos, estabilidad y
respuesta, está por encima de lo generalmente aceptado por los
estándares en tiempo de respuesta de aplicaciones de software
empresarial, en este caso particular, de tipo web para la industria de
tecnología en inversión financiera, fintech.

\begin{quote}
10 seconds is about the limit for keeping the user's attention focused
on the dialogue. For longer delays, users will want to perform other
tasks while waiting for the computer to finish, so they should be given
feedback indicating when the computer expects to be done. Feedback
during the delay is especially important if the response time is likely
to be highly variable, since users will then not know what to expect. --
Nielsen, J. (1993). Usability Engineering. Response Times: The 3
Important Limits (web).
\end{quote}

\subsubsection{Conclusión General}\label{sec:conclusiuxf3n-general}

Teniendo de base los resultados de la actual prueba de rendimiento
consignados en el Informe Técnico de Certificación Operativa Plataforma
de Software Trii.co, es factible indicar que el umbral de crecimiento de
la Plataforma Trii, sin que alcance a comprometer la estabilidad de la
Aplicación, en términos de nivel de ocupación de recursos y tasa de
éxito, podría llegar a ser de entre 4x y 5x de la carga de procesamiento
real actual. Es decir, con la capacidad operativa actual, sin requerir
inversión en su plan de capacidad, podría aumentar sus niveles de
procesamiento en un 400\% (esto es, de \textasciitilde5000 transacciones
diarias a 22774), como mínimo, sin comprometer la estabilidad del
sistema completo.

\end{document}
