%%
% Copyright (c) 2017 - 2024, Pascal Wagler;
% Copyright (c) 2014 - 2024, John MacFarlane
%
% All rights reserved.
%
% Redistribution and use in source and binary forms, with or without
% modification, are permitted provided that the following conditions
% are met:
%
% - Redistributions of source code must retain the above copyright
% notice, this list of conditions and the following disclaimer.
%
% - Redistributions in binary form must reproduce the above copyright
% notice, this list of conditions and the following disclaimer in the
% documentation and/or other materials provided with the distribution.
%
% - Neither the name of John MacFarlane nor the names of other
% contributors may be used to endorse or promote products derived
% from this software without specific prior written permission.
%
% THIS SOFTWARE IS PROVIDED BY THE COPYRIGHT HOLDERS AND CONTRIBUTORS
% "AS IS" AND ANY EXPRESS OR IMPLIED WARRANTIES, INCLUDING, BUT NOT
% LIMITED TO, THE IMPLIED WARRANTIES OF MERCHANTABILITY AND FITNESS
% FOR A PARTICULAR PURPOSE ARE DISCLAIMED. IN NO EVENT SHALL THE
% COPYRIGHT OWNER OR CONTRIBUTORS BE LIABLE FOR ANY DIRECT, INDIRECT,
% INCIDENTAL, SPECIAL, EXEMPLARY, OR CONSEQUENTIAL DAMAGES (INCLUDING,
% BUT NOT LIMITED TO, PROCUREMENT OF SUBSTITUTE GOODS OR SERVICES;
% LOSS OF USE, DATA, OR PROFITS; OR BUSINESS INTERRUPTION) HOWEVER
% CAUSED AND ON ANY THEORY OF LIABILITY, WHETHER IN CONTRACT, STRICT
% LIABILITY, OR TORT (INCLUDING NEGLIGENCE OR OTHERWISE) ARISING IN
% ANY WAY OUT OF THE USE OF THIS SOFTWARE, EVEN IF ADVISED OF THE
% POSSIBILITY OF SUCH DAMAGE.
%%

%%
% This is the Eisvogel pandoc LaTeX template.
%
% For usage information and examples visit the official GitHub page:
% https://github.com/Wandmalfarbe/pandoc-latex-template
%%

% Options for packages loaded elsewhere
\PassOptionsToPackage{unicode}{hyperref}
\PassOptionsToPackage{hyphens}{url}
\PassOptionsToPackage{dvipsnames,svgnames,x11names,table}{xcolor}
%
\documentclass[
  paper=a4,
  ,captions=tableheading
]{scrartcl}
\usepackage{amsmath,amssymb}
% Use setspace anyway because we change the default line spacing.
% The spacing is changed early to affect the titlepage and the TOC.
\usepackage{setspace}
\setstretch{1.2}
\usepackage{iftex}
\ifPDFTeX
  \usepackage[T1]{fontenc}
  \usepackage[utf8]{inputenc}
  \usepackage{textcomp} % provide euro and other symbols
\else % if luatex or xetex
  \usepackage{unicode-math} % this also loads fontspec
  \defaultfontfeatures{Scale=MatchLowercase}
  \defaultfontfeatures[\rmfamily]{Ligatures=TeX,Scale=1}
\fi
\usepackage{lmodern}
\ifPDFTeX\else
  % xetex/luatex font selection
\fi
% Use upquote if available, for straight quotes in verbatim environments
\IfFileExists{upquote.sty}{\usepackage{upquote}}{}
\IfFileExists{microtype.sty}{% use microtype if available
  \usepackage[]{microtype}
  \UseMicrotypeSet[protrusion]{basicmath} % disable protrusion for tt fonts
}{}
\makeatletter
\@ifundefined{KOMAClassName}{% if non-KOMA class
  \IfFileExists{parskip.sty}{%
    \usepackage{parskip}
  }{% else
    \setlength{\parindent}{0pt}
    \setlength{\parskip}{6pt plus 2pt minus 1pt}}
}{% if KOMA class
  \KOMAoptions{parskip=half}}
\makeatother
\usepackage{xcolor}
\definecolor{default-linkcolor}{HTML}{A50000}
\definecolor{default-filecolor}{HTML}{A50000}
\definecolor{default-citecolor}{HTML}{4077C0}
\definecolor{default-urlcolor}{HTML}{4077C0}
\usepackage[top=1.3in,bottom=1in,left=0.7in,right=0.7in]{geometry}
\usepackage[export]{adjustbox}
\usepackage{graphicx}
\usepackage{etoolbox}
\BeforeBeginEnvironment{lstlisting}{\par\noindent\begin{minipage}{\linewidth}}
\AfterEndEnvironment{lstlisting}{\end{minipage}\par\addvspace{\topskip}}
\usepackage{longtable,booktabs,array}
\usepackage{calc} % for calculating minipage widths
% Correct order of tables after \paragraph or \subparagraph
\usepackage{etoolbox}
\makeatletter
\patchcmd\longtable{\par}{\if@noskipsec\mbox{}\fi\par}{}{}
\makeatother
% Allow footnotes in longtable head/foot
\IfFileExists{footnotehyper.sty}{\usepackage{footnotehyper}}{\usepackage{footnote}}
\makesavenoteenv{longtable}
% add backlinks to footnote references, cf. https://tex.stackexchange.com/questions/302266/make-footnote-clickable-both-ways
\usepackage{footnotebackref}
\setlength{\emergencystretch}{3em} % prevent overfull lines
\providecommand{\tightlist}{%
  \setlength{\itemsep}{0pt}\setlength{\parskip}{0pt}}
\setcounter{secnumdepth}{-\maxdimen} % remove section numbering
\ifLuaTeX
\usepackage[bidi=basic]{babel}
\else
\usepackage[bidi=default]{babel}
\fi
\babelprovide[main,import]{english}
% get rid of language-specific shorthands (see #6817):
\let\LanguageShortHands\languageshorthands
\def\languageshorthands#1{}
\makeatletter
\@ifpackageloaded{subfig}{}{\usepackage{subfig}}
\@ifpackageloaded{caption}{}{\usepackage{caption}}
\captionsetup[subfloat]{margin=0.5em}
\AtBeginDocument{%
\renewcommand*\figurename{Figura}
\renewcommand*\tablename{Tabla}
}
\AtBeginDocument{%
\renewcommand*\listfigurename{Lista de Figuras}
\renewcommand*\listtablename{Lista de Tablas}
}
\newcounter{pandoccrossref@subfigures@footnote@counter}
\newenvironment{pandoccrossrefsubfigures}{%
\setcounter{pandoccrossref@subfigures@footnote@counter}{0}
\begin{figure}\centering%
\gdef\global@pandoccrossref@subfigures@footnotes{}%
\DeclareRobustCommand{\footnote}[1]{\footnotemark%
\stepcounter{pandoccrossref@subfigures@footnote@counter}%
\ifx\global@pandoccrossref@subfigures@footnotes\empty%
\gdef\global@pandoccrossref@subfigures@footnotes{{##1}}%
\else%
\g@addto@macro\global@pandoccrossref@subfigures@footnotes{, {##1}}%
\fi}}%
{\end{figure}%
\addtocounter{footnote}{-\value{pandoccrossref@subfigures@footnote@counter}}
\@for\f:=\global@pandoccrossref@subfigures@footnotes\do{\stepcounter{footnote}\footnotetext{\f}}%
\gdef\global@pandoccrossref@subfigures@footnotes{}}
\@ifpackageloaded{float}{}{\usepackage{float}}
\floatstyle{ruled}
\@ifundefined{c@chapter}{\newfloat{codelisting}{h}{lop}}{\newfloat{codelisting}{h}{lop}[chapter]}
\floatname{codelisting}{Listing}
\newcommand*\listoflistings{\listof{codelisting}{Listas del Documento}}
\makeatother
\usepackage{bookmark}
\IfFileExists{xurl.sty}{\usepackage{xurl}}{} % add URL line breaks if available
\urlstyle{same}
\hypersetup{
  pdftitle={Certificación Operativa Plataforma de Software trii},
  pdfauthor={SoftProductiva.com},
  pdflang={en},
  pdfsubject={Implementación Proyecto},
  pdfkeywords={Rendimiento, Métodos pruebas, Pruebas software, QA},
  hidelinks,
  breaklinks=true,
  pdfcreator={LaTeX via pandoc with the Eisvogel template}}
\title{Certificación Operativa Plataforma de Software trii}
\usepackage{etoolbox}
\makeatletter
\providecommand{\subtitle}[1]{% add subtitle to \maketitle
  \apptocmd{\@title}{\par {\large #1 \par}}{}{}
}
\makeatother
\subtitle{Línea Base}
\author{SoftProductiva.com}
\date{2025-01-20}



%%
%% added
%%


%
% for the background color of the title page
%
\usepackage{pagecolor}
\usepackage{afterpage}
\usepackage{tikz}

%
% break urls
%
\PassOptionsToPackage{hyphens}{url}

%
% When using babel or polyglossia with biblatex, loading csquotes is recommended
% to ensure that quoted texts are typeset according to the rules of your main language.
%
\usepackage{csquotes}

%
% captions
%
\definecolor{caption-color}{HTML}{777777}
\usepackage[font={stretch=1.2}, textfont={color=caption-color}, position=top, skip=4mm, labelfont=bf, singlelinecheck=false, justification=raggedright]{caption}
\setcapindent{0em}

%
% blockquote
%
\definecolor{blockquote-border}{RGB}{221,221,221}
\definecolor{blockquote-text}{RGB}{119,119,119}
\usepackage{mdframed}
\newmdenv[rightline=false,bottomline=false,topline=false,linewidth=3pt,linecolor=blockquote-border,skipabove=\parskip]{customblockquote}
\renewenvironment{quote}{\begin{customblockquote}\list{}{\rightmargin=0em\leftmargin=0em}%
\item\relax\color{blockquote-text}\ignorespaces}{\unskip\unskip\endlist\end{customblockquote}}

%
% Source Sans Pro as the default font family
% Source Code Pro for monospace text
%
% 'default' option sets the default
% font family to Source Sans Pro, not \sfdefault.
%
\ifnum 0\ifxetex 1\fi\ifluatex 1\fi=0 % if pdftex
    \usepackage[default]{sourcesanspro}
  \usepackage{sourcecodepro}
  \else % if not pdftex
    \usepackage[default]{sourcesanspro}
  \usepackage{sourcecodepro}

  % XeLaTeX specific adjustments for straight quotes: https://tex.stackexchange.com/a/354887
  % This issue is already fixed (see https://github.com/silkeh/latex-sourcecodepro/pull/5) but the
  % fix is still unreleased.
  % TODO: Remove this workaround when the new version of sourcecodepro is released on CTAN.
  \ifxetex
    \makeatletter
    \defaultfontfeatures[\ttfamily]
      { Numbers   = \sourcecodepro@figurestyle,
        Scale     = \SourceCodePro@scale,
        Extension = .otf }
    \setmonofont
      [ UprightFont    = *-\sourcecodepro@regstyle,
        ItalicFont     = *-\sourcecodepro@regstyle It,
        BoldFont       = *-\sourcecodepro@boldstyle,
        BoldItalicFont = *-\sourcecodepro@boldstyle It ]
      {SourceCodePro}
    \makeatother
  \fi
  \fi

%
% heading color
%
\definecolor{heading-color}{RGB}{40,40,40}
\addtokomafont{section}{\color{heading-color}}
% When using the classes report, scrreprt, book,
% scrbook or memoir, uncomment the following line.
%\addtokomafont{chapter}{\color{heading-color}}

%
% variables for title, author and date
%
\usepackage{titling}
\title{Certificación Operativa Plataforma de Software trii}
\author{SoftProductiva.com}
\date{2025-01-20}

%
% tables
%

\definecolor{table-row-color}{HTML}{F5F5F5}
\definecolor{table-rule-color}{HTML}{999999}

%\arrayrulecolor{black!40}
\arrayrulecolor{table-rule-color}     % color of \toprule, \midrule, \bottomrule
\setlength\heavyrulewidth{0.3ex}      % thickness of \toprule, \bottomrule
\renewcommand{\arraystretch}{0.8}     % spacing (padding)


%
% remove paragraph indentation
%
\setlength{\parindent}{0pt}
\setlength{\parskip}{6pt plus 2pt minus 1pt}
\setlength{\emergencystretch}{3em}  % prevent overfull lines

%
%
% Listings
%
%


%
% header and footer
%
\usepackage[headsepline,footsepline]{scrlayer-scrpage}

\newpairofpagestyles{eisvogel-header-footer}{
  \clearpairofpagestyles
  \ihead*{2025-01-20}
  \chead*{}
  \ohead*{Certificación Operativa Plataforma de Software trii}
  \ifoot*{SoftProductiva.com}
  \cfoot*{}
  \ofoot*{\thepage}
  \addtokomafont{pageheadfoot}{\upshape}
}
\pagestyle{eisvogel-header-footer}



%%
%% end added
%%

%% change the default with of the images using \setkeys{Gin}{width=.8\linewidth}
\setkeys{Gin}{width=1\linewidth}

\begin{document}

%%
%% begin titlepage
%%
\begin{titlepage}
\newgeometry{top=2cm, right=4cm, bottom=3cm, left=4cm}
\tikz[remember picture,overlay] \node[inner sep=0pt] at (current page.center){\includegraphics[width=\paperwidth,height=\paperheight]{include/background.pdf}};
\newcommand{\colorRule}[3][black]{\textcolor[HTML]{#1}{\rule{#2}{#3}}}
\begin{flushleft}
\noindent
\\[-1em]
\color[HTML]{5F5F5F}
\makebox[0pt][l]{\colorRule[360049]{1.3\textwidth}{4pt}}
\par
\noindent

% The titlepage with a background image has other text spacing and text size
{
  \setstretch{2}
  \vfill
  \vskip -8em
  \noindent {\huge \textbf{\textsf{Certificación Operativa Plataforma de
Software trii}}}
    \vskip 1em
  {\Large \textsf{Línea Base}}
    \vskip 2em
  \noindent {\Large \textsf{SoftProductiva.com} \vskip 0.6em \textsf{2025-01-20}}
  \vfill
}

\noindent
\includegraphics[width=60mm, right]{include/logo.png}


\end{flushleft}
\end{titlepage}
\restoregeometry
\pagenumbering{arabic}

%%
%% end titlepage
%%

% \maketitle


\section{Contenido}\label{sec:contenido}

\begin{itemize}
\tightlist
\item
  \hyperref[informaciuxf3n-del-documento]{Información del Documento}
\item
  \hyperref[luxednea-base-del-sistemaaplicaciuxf3n]{Línea Base del
  Sistema/Aplicación}
\end{itemize}

\newpage

\section{Información del
Documento}\label{sec:informaciuxf3n-del-documento}

\subsection{Versión del Documento}\label{sec:versiuxf3n-del-documento}

\begin{quote}
\end{quote}

\subsection{Control de Cambios}\label{sec:control-de-cambios}

Historia de cambios del informe.

Versión actual: 1.b606138 - Compilación para entrega: 1.5b6d694.3674428:
revision - Fri, 24 Jan 2025 23:07:07 +0000

Versiones Anteriores

1.b606138 - Compilación para entrega: 1.5b6d694.3674428: revision - Fri,
24 Jan 2025 23:07:07 +0000

1.5b6d694 - Compilación para entrega: 1.294ee2b.d30c61d: notasalpie -
Fri, 24 Jan 2025 23:02:57 +0000

1.294ee2b - Compilación para entrega: 1.6b2bb58.f6a8199: formato - Fri,
24 Jan 2025 23:01:49 +0000

1.6b2bb58 - build - Fri, 24 Jan 2025 18:01:03 -0500

\subsubsection{Realizado Por}\label{sec:realizado-por}

H. Wong, ing.

\subsubsection{Revisado Por}\label{sec:revisado-por}

(revisor), trii

\newpage

\section{Línea Base del
Sistema/Aplicación}\label{sec:luxednea-base-del-sistemaaplicaciuxf3n}

\subsection{Componentes del Informe de Rendimiento y Capacidad de la
Plataforma
trii}\label{sec:componentes-del-informe-de-rendimiento-y-capacidad-de-la-plataforma-trii}

\begin{quote}
\end{quote}

\subsubsection{Información General de la Línea
Base}\label{sec:informaciuxf3n-general-de-la-luxednea-base}

\begin{itemize}
\tightlist
\item
  Nombre de la Aplicación/Sistema Probado: Servicios de Órdenes, Auth, y
  User Info de la Aplicación trii
\item
  Versión de la Aplicación/Sistema: Versión 2025
\item
  Entorno de Pruebas: infraestructura en la nube, Google Cloud, 2nd
  generation machine series, General-purpose workloads E2 serie, CPU
  Intel. Tipo de equipo: highmem, 7-14 GB.
\item
  Fecha/Periodo de Pruebas: 15 de enero del 2025.
\item
  Objetivos de las Pruebas:

  \begin{itemize}
  \tightlist
  \item
    Encontrar la capacidad de los servicios Servicios Órdenes, Auth, y
    User Info de la Aplicación por separado en número máximo de
    operaciones o transacciones de los servicios por unidad de tiempo.
  \item
    Encontrar el nivel de estabilidad de los servicios Servicios
    Órdenes, Auth, y User Info (tensión) de la Aplicación.
  \item
    Dar pautas alrededor del estrés o tensión de los servicios Servicios
    Órdenes, Auth, y User Info de la Aplicación por separado para
    determinar la holgura respecto a la demanda esperada.
  \end{itemize}
\item
  Métricas Clave:

  \begin{itemize}
  \tightlist
  \item
    Capacidad (throughput) de los servicios Servicios Órdenes, Auth, y
    User Info
  \item
    Estrés (tensión) de los servicios Servicios Órdenes, Auth, y User
    Info
  \item
    Estabilidad (Uso de CPU) de los servicios Servicios Órdenes, Auth, y
    User Info Herramienta de Pruebas: K6, de Grafana Labs.
  \end{itemize}
\end{itemize}

\subsubsection{Línea Base Servicio Get User Info de
trii}\label{sec:luxednea-base-servicio-get-user-info-de-trii}

El servicio Get User Info (user info) obtiene datos de trabajo del
cliente previo a la orden. Requiere como mínimo actividades de
autenticación, y es responsable de alimentar al servicio Órdenes.

\paragraph{Valores Numéricos}\label{sec:valores-numuxe9ricos}

En condiciones operativas usuales, promedio por transacción, tiempo
máximo, mínimo, y percentiles de las métricas. Tomado del mayor entre
http\_req\_duration e iteration\_duration.

\begin{quote}
Tiempo máximo de la transacción (iteración): max=1.638s

Tiempo promedio: avg=460.499ms

Tiempo mínimo: min=190.9895ms

Percentil 90 duración iteración: p(90)=641.992ms

Cantidad de transacciones/segundo (capacidad o throughput): 641.992/s

Estabilidad o Tasa de éxito de transacción (promedio entre dos
servicios, login y user\_info): 100.00\%

Latencia promedio: avg=227.331ms

Latencia máxima: max=1.5535s
\end{quote}

\subsubsection{Línea Base Servicio Servicio Login Auth de
trii}\label{sec:luxednea-base-servicio-servicio-login-auth-de-trii}

El servicio Login (auth) es responsable de dar inicio a una sesión de
trabajo de un cliente trii. Realiza como mínimo la provisión de datos
necesarios a otros servicios respecto de la verificación y creación de
una sesión de trabajo válida.

\paragraph{Valores Numéricos}\label{sec:valores-numuxe9ricos-1}

En condiciones operativas usuales, promedio por transacción, tiempo
máximo, mínimo, y percentiles de las métricas. Tomado del mayor entre
http\_req\_duration e iteration\_duration.

\begin{quote}
Tiempo máximo de la transacción (iteración): max=2.3855s

Tiempo promedio: avg=115.297ms

Tiempo mínimo: min=68.7115ms

Percentil 90 duración iteración: p(90)=170.196ms

Cantidad de transacciones/segundo (capacidad o throughput): 122.975268/s

Estabilidad o Tasa de éxito de transacción: 100.00\%

Latencia promedio: avg=115.037ms

Latencia máxima: max=205.088ms
\end{quote}

\subsubsection{Línea Base Servicio Órdenes de
trii}\label{sec:luxednea-base-servicio-uxf3rdenes-de-trii}

El servicio Órdenes es el más relevante para el negocio de trii. Realiza
como mínimo actividades de creación de una orden de negocio, que es la
entidad de información superlativa de la plataforma.

\paragraph{Valores Numéricos}\label{sec:valores-numuxe9ricos-2}

En condiciones operativas usuales, promedio por transacción, tiempo
máximo, mínimo, y percentiles de las métricas. Tomado del mayor entre
http\_req\_duration e iteration\_duration.

\begin{quote}
Tiempo máximo de la transacción (iteración): max=2.9185s

Tiempo promedio: avg=2.1515s

Tiempo mínimo: min=272.844ms

Percentil 90 duración iteración: p(90)=1.8395s

Cantidad de transacciones/segundo (capacidad o throughput): 10.637276/s

Estabilidad o Tasa de éxito de transacción (iteración): 100.00\%

Latencia promedio: avg=577.408ms

Latencia máxima: max=2.47s
\end{quote}

\end{document}
