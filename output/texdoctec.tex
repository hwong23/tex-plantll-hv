%%
% Copyright (c) 2017 - 2024, Pascal Wagler;
% Copyright (c) 2014 - 2024, John MacFarlane
%
% All rights reserved.
%
% Redistribution and use in source and binary forms, with or without
% modification, are permitted provided that the following conditions
% are met:
%
% - Redistributions of source code must retain the above copyright
% notice, this list of conditions and the following disclaimer.
%
% - Redistributions in binary form must reproduce the above copyright
% notice, this list of conditions and the following disclaimer in the
% documentation and/or other materials provided with the distribution.
%
% - Neither the name of John MacFarlane nor the names of other
% contributors may be used to endorse or promote products derived
% from this software without specific prior written permission.
%
% THIS SOFTWARE IS PROVIDED BY THE COPYRIGHT HOLDERS AND CONTRIBUTORS
% "AS IS" AND ANY EXPRESS OR IMPLIED WARRANTIES, INCLUDING, BUT NOT
% LIMITED TO, THE IMPLIED WARRANTIES OF MERCHANTABILITY AND FITNESS
% FOR A PARTICULAR PURPOSE ARE DISCLAIMED. IN NO EVENT SHALL THE
% COPYRIGHT OWNER OR CONTRIBUTORS BE LIABLE FOR ANY DIRECT, INDIRECT,
% INCIDENTAL, SPECIAL, EXEMPLARY, OR CONSEQUENTIAL DAMAGES (INCLUDING,
% BUT NOT LIMITED TO, PROCUREMENT OF SUBSTITUTE GOODS OR SERVICES;
% LOSS OF USE, DATA, OR PROFITS; OR BUSINESS INTERRUPTION) HOWEVER
% CAUSED AND ON ANY THEORY OF LIABILITY, WHETHER IN CONTRACT, STRICT
% LIABILITY, OR TORT (INCLUDING NEGLIGENCE OR OTHERWISE) ARISING IN
% ANY WAY OUT OF THE USE OF THIS SOFTWARE, EVEN IF ADVISED OF THE
% POSSIBILITY OF SUCH DAMAGE.
%%

%%
% This is the Eisvogel pandoc LaTeX template.
%
% For usage information and examples visit the official GitHub page:
% https://github.com/Wandmalfarbe/pandoc-latex-template
%%

% Options for packages loaded elsewhere
\PassOptionsToPackage{unicode}{hyperref}
\PassOptionsToPackage{hyphens}{url}
\PassOptionsToPackage{dvipsnames,svgnames,x11names,table}{xcolor}
%
\documentclass[
  paper=a4,
  ,captions=tableheading
]{scrartcl}
\usepackage{amsmath,amssymb}
% Use setspace anyway because we change the default line spacing.
% The spacing is changed early to affect the titlepage and the TOC.
\usepackage{setspace}
\setstretch{1.2}
\usepackage{iftex}
\ifPDFTeX
  \usepackage[T1]{fontenc}
  \usepackage[utf8]{inputenc}
  \usepackage{textcomp} % provide euro and other symbols
\else % if luatex or xetex
  \usepackage{unicode-math} % this also loads fontspec
  \defaultfontfeatures{Scale=MatchLowercase}
  \defaultfontfeatures[\rmfamily]{Ligatures=TeX,Scale=1}
\fi
\usepackage{lmodern}
\ifPDFTeX\else
  % xetex/luatex font selection
\fi
% Use upquote if available, for straight quotes in verbatim environments
\IfFileExists{upquote.sty}{\usepackage{upquote}}{}
\IfFileExists{microtype.sty}{% use microtype if available
  \usepackage[]{microtype}
  \UseMicrotypeSet[protrusion]{basicmath} % disable protrusion for tt fonts
}{}
\makeatletter
\@ifundefined{KOMAClassName}{% if non-KOMA class
  \IfFileExists{parskip.sty}{%
    \usepackage{parskip}
  }{% else
    \setlength{\parindent}{0pt}
    \setlength{\parskip}{6pt plus 2pt minus 1pt}}
}{% if KOMA class
  \KOMAoptions{parskip=half}}
\makeatother
\usepackage{xcolor}
\definecolor{default-linkcolor}{HTML}{A50000}
\definecolor{default-filecolor}{HTML}{A50000}
\definecolor{default-citecolor}{HTML}{4077C0}
\definecolor{default-urlcolor}{HTML}{4077C0}
\usepackage[margin=2.5cm,includehead=true,includefoot=true,centering,]{geometry}
% add backlinks to footnote references, cf. https://tex.stackexchange.com/questions/302266/make-footnote-clickable-both-ways
\usepackage{footnotebackref}
\usepackage{graphicx}
\makeatletter
\newsavebox\pandoc@box
\newcommand*\pandocbounded[1]{% scales image to fit in text height/width
  \sbox\pandoc@box{#1}%
  \Gscale@div\@tempa{\textheight}{\dimexpr\ht\pandoc@box+\dp\pandoc@box\relax}%
  \Gscale@div\@tempb{\linewidth}{\wd\pandoc@box}%
  \ifdim\@tempb\p@<\@tempa\p@\let\@tempa\@tempb\fi% select the smaller of both
  \ifdim\@tempa\p@<\p@\scalebox{\@tempa}{\usebox\pandoc@box}%
  \else\usebox{\pandoc@box}%
  \fi%
}
% Set default figure placement to htbp
% Make use of float-package and set default placement for figures to H.
% The option H means 'PUT IT HERE' (as  opposed to the standard h option which means 'You may put it here if you like').
\usepackage{float}
\floatplacement{figure}{H}
\makeatother
\setlength{\emergencystretch}{3em} % prevent overfull lines
\providecommand{\tightlist}{%
  \setlength{\itemsep}{0pt}\setlength{\parskip}{0pt}}
\setcounter{secnumdepth}{-\maxdimen} % remove section numbering
\ifLuaTeX
\usepackage[bidi=basic]{babel}
\else
\usepackage[bidi=default]{babel}
\fi
\babelprovide[main,import]{english}
% get rid of language-specific shorthands (see #6817):
\let\LanguageShortHands\languageshorthands
\def\languageshorthands#1{}
\usepackage{bookmark}
\IfFileExists{xurl.sty}{\usepackage{xurl}}{} % add URL line breaks if available
\urlstyle{same}
\hypersetup{
  pdftitle={Example PDF},
  pdfauthor={Author},
  pdflang={en},
  pdfsubject={Markdown},
  pdfkeywords={Markdown, Example},
  hidelinks,
  breaklinks=true,
  pdfcreator={LaTeX via pandoc with the Eisvogel template}}
\title{Example PDF}
\author{Author}
\date{2017-02-20}



%%
%% added
%%


%
% for the background color of the title page
%

%
% break urls
%
\PassOptionsToPackage{hyphens}{url}

%
% When using babel or polyglossia with biblatex, loading csquotes is recommended
% to ensure that quoted texts are typeset according to the rules of your main language.
%
\usepackage{csquotes}

%
% captions
%
\definecolor{caption-color}{HTML}{777777}
\usepackage[font={stretch=1.2}, textfont={color=caption-color}, position=top, skip=4mm, labelfont=bf, singlelinecheck=false, justification=raggedright]{caption}
\setcapindent{0em}

%
% blockquote
%
\definecolor{blockquote-border}{RGB}{221,221,221}
\definecolor{blockquote-text}{RGB}{119,119,119}
\usepackage{mdframed}
\newmdenv[rightline=false,bottomline=false,topline=false,linewidth=3pt,linecolor=blockquote-border,skipabove=\parskip]{customblockquote}
\renewenvironment{quote}{\begin{customblockquote}\list{}{\rightmargin=0em\leftmargin=0em}%
\item\relax\color{blockquote-text}\ignorespaces}{\unskip\unskip\endlist\end{customblockquote}}

%
% Source Sans Pro as the default font family
% Source Code Pro for monospace text
%
% 'default' option sets the default
% font family to Source Sans Pro, not \sfdefault.
%
\ifnum 0\ifxetex 1\fi\ifluatex 1\fi=0 % if pdftex
    \usepackage[default]{sourcesanspro}
  \usepackage{sourcecodepro}
  \else % if not pdftex
    \usepackage[default]{sourcesanspro}
  \usepackage{sourcecodepro}

  % XeLaTeX specific adjustments for straight quotes: https://tex.stackexchange.com/a/354887
  % This issue is already fixed (see https://github.com/silkeh/latex-sourcecodepro/pull/5) but the
  % fix is still unreleased.
  % TODO: Remove this workaround when the new version of sourcecodepro is released on CTAN.
  \ifxetex
    \makeatletter
    \defaultfontfeatures[\ttfamily]
      { Numbers   = \sourcecodepro@figurestyle,
        Scale     = \SourceCodePro@scale,
        Extension = .otf }
    \setmonofont
      [ UprightFont    = *-\sourcecodepro@regstyle,
        ItalicFont     = *-\sourcecodepro@regstyle It,
        BoldFont       = *-\sourcecodepro@boldstyle,
        BoldItalicFont = *-\sourcecodepro@boldstyle It ]
      {SourceCodePro}
    \makeatother
  \fi
  \fi

%
% heading color
%
\definecolor{heading-color}{RGB}{40,40,40}
\addtokomafont{section}{\color{heading-color}}
% When using the classes report, scrreprt, book,
% scrbook or memoir, uncomment the following line.
%\addtokomafont{chapter}{\color{heading-color}}

%
% variables for title, author and date
%
\usepackage{titling}
\title{Example PDF}
\author{Author}
\date{2017-02-20}

%
% tables
%

%
% remove paragraph indentation
%
\setlength{\parindent}{0pt}
\setlength{\parskip}{6pt plus 2pt minus 1pt}
\setlength{\emergencystretch}{3em}  % prevent overfull lines

%
%
% Listings
%
%


%
% header and footer
%
\usepackage[headsepline,footsepline]{scrlayer-scrpage}

\newpairofpagestyles{eisvogel-header-footer}{
  \clearpairofpagestyles
  \ihead*{Example PDF}
  \chead*{}
  \ohead*{2017-02-20}
  \ifoot*{Author}
  \cfoot*{}
  \ofoot*{\thepage}
  \addtokomafont{pageheadfoot}{\upshape}
}
\pagestyle{eisvogel-header-footer}



%%
%% end added
%%

\begin{document}

%%
%% begin titlepage
%%

%%
%% end titlepage
%%

% \maketitle


\hypertarget{reqr.1n.-requerimientos}{%
\section{05.REQR.1n. Requerimientos}\label{reqr.1n.-requerimientos}}

\begin{itemize}
\tightlist
\item
  \protect\hyperlink{Introducciuxf3n}{Introducción}
\item
  \protect\hyperlink{problema-2-grouping}{Problema 2 (Grouping)}

  \begin{itemize}
  \tightlist
  \item
    \protect\hyperlink{aceptaciuxf3n-value}{Aceptación (Value)}
  \item
    \protect\hyperlink{complejidad-constraint}{Complejidad (Constraint)}
  \item
    \protect\hyperlink{no-funcional-goal}{No Funcional (Goal)}
  \end{itemize}
\item
  \protect\hyperlink{soluciuxf3n-2-grouping}{Solución 2 (Grouping)}

  \begin{itemize}
  \tightlist
  \item
    \protect\hyperlink{technology-service-technology-service}{Technology
    Service (Technology Service)}
  \end{itemize}
\item
  \protect\hyperlink{soluciuxf3n-1-grouping}{Solución 1 (Grouping)}

  \begin{itemize}
  \tightlist
  \item
    \protect\hyperlink{sint1.-integraciuxf3n.-ingreso-a-conti-application-service}{SINT1.
    Integración. Ingreso a Conti (Application Service)}
  \item
    \protect\hyperlink{sint2.-integraciuxf3n.-consulta-uxedtem-de-conti-application-service}{SINT2.
    Integración. Consulta ítem de Conti (Application Service)}
  \item
    \protect\hyperlink{sint3.-integraciuxf3n.-radicar-uxedtem-application-service}{SINT3.
    Integración. Radicar ítem (Application Service)}
  \item
    \protect\hyperlink{sint4.-integraciuxf3n.-generaciuxf3n-de-documentos-application-service}{SINT4.
    Integración. Generación de documentos (Application Service)}
  \end{itemize}
\item
  \protect\hyperlink{problema-1-grouping}{Problema 1 (Grouping)}

  \begin{itemize}
  \tightlist
  \item
    \protect\hyperlink{levantamiento-constraint}{Levantamiento
    (Constraint)}
  \item
    \protect\hyperlink{contractual-goal}{Contractual (Goal)}
  \item
    \protect\hyperlink{entrega-value}{Entrega (Value)}
  \end{itemize}
\item
  \protect\hyperlink{reqr3.-integraciuxf3n-con-sistema-conti-requirement}{REQR3.
  Integración con Sistema Conti (Requirement)}
\item
  \protect\hyperlink{reqr2.-condiciones-tecnoluxf3gicas-jep-requirement}{REQR2.
  Condiciones tecnológicas JEP (Requirement)}
\end{itemize}

\hypertarget{introducciuxf3n}{%
\subsection{Introducción}\label{introducciuxf3n}}

\begin{figure}
\centering
\caption{05.REQR.1n. Requerimientos}
\end{figure}

Para la implementación de los ítems relacionados en el Anexo Nro. 1.1 --
Anexo técnico evolución plataforma de interoperabilidad -- Ficha Técnica
la hoja ``Categorías de Cotización'' contiene las necesidades a
contratar en el ámbito de la evolución tecnológica del modelo de
interoperabilidad y los desarrollos de interoperabilidad tanto con
sistemas internos, como con entidades externas. En la hoja ``Estándares
Desarrollo y Producto'' del archivo mencionado se indican los estándares
recomendados por el fabricante, para tener en cuenta en la entrega de
los servicios que se cotizan.

El Anexo Nro. 1.2 -- Acuerdos de Niveles de Servicio, explica el
procedimiento con el que se dará atención a consultas o solución de
incidencias, tanto en los sistemas operativos, como en los servicios de
interoperabilidad existentes en la actualidad y aquellos que se
contratarán en este proceso, en el sistema Bus de Interoperabilidad
implementado en la Jurisdicción Especial para la Paz.

-- Documento: JUSTIFICATIVO DE LA CONTRATACIÓN INVITACIÓN PÚBLICA

\hypertarget{problema-2}{%
\subsection{Problema 2}\label{problema-2}}

\hypertarget{aceptaciuxf3n}{%
\subsubsection{Aceptación}\label{aceptaciuxf3n}}

\hypertarget{complejidad}{%
\subsubsection{Complejidad}\label{complejidad}}

\hypertarget{no-funcional}{%
\subsubsection{No Funcional}\label{no-funcional}}

\begin{enumerate}
\def\labelenumi{\arabic{enumi}.}
\tightlist
\item
  La solución debe implementar el habilitar la traducción de múltiples
  protocolos del consumidor a un protocolo específico del microservicio
  ofrecido a través de un API Gateway
\item
  La solución debe implementar la publicación de microservicios que
  generen múltiples API para plataformas y clientes específicos con las
  funciones específicas y protocolos requeridos por cada plataforma
\item
  La solución debe implementar un mecanismo de hacer la trazabilidad,
  uso y registro de actividades de los microservicios.
\item
  Debe permitir la integración con un servicio de directorio corporativo
  que puede servir como administrador de identidad corporativa. Por lo
  tanto, la solución debe poder actuar como Administrador de acceso
  (Identity Access Manager - IaM) mientras que el Servicio de directorio
  sirve como Administrador e Identidades (Identity Manager - IdM).
\item
  Debe admitir la transformación y el enrutamiento hacia / desde SOAP /
  HTTP a los servicios REST.
\item
  Debe poder convertir mensajes a / desde: XML, objetos Java, JSON,
  REST, CSV.
\item
  Debe proporcionar componentes para la transformación utilizando
  modelos predefinidos (plantillas).
\item
  La infraestructura debe distribuirse de modo que las integraciones,
  construidas a partir de EIP (patrones de integración empresarial) y
  conectores predefinidos, se implementen en la infraestructura nativa
  del contenedor para adaptarse y escalar rápidamente
\item
  La solución debería exponer métricas con integración nativa al
  software Prometheus
\item
  Debe ser posible gestionar la creación de un token de acceso,
  eligiendo su alcance, permiso y otras cualidades a nivel de
  autenticación
\item
  La solución de administración de API debe admitir la instalación de
  API Gateways en contenedores tanto dentro de una plataforma Kubernetes
  o utilizando en motor de contenedor aprobado por la especificación OCI
\item
  Debe permitir ver las llamadas a la API y separar los códigos de
  retorno HTTP;
\item
  La herramienta API Gateway debe controlar la ejecución de llamadas,
  recopilar métricas, aplicar políticas y límites de ejecución;
\item
  Para permitir interoperabilidad debe habilitar transporte de mensajes
  y conectarse entre ellos. Los mecanismos de transporte deben incluir
  Java Messaging Service (JMS), Active MQ y asi mismo protocolos de
  comunicación tal como HTTP/HTTPS,SMTP, entre otros
\item
  Se debe contar con la característica de Single Sign On (SSO)
\item
  Los servicios se deberán implementar bajo la plataforma Openshift de
  RedHat
\item
  Se debe contemplar dentro de estos desarrollos la Transferencia de
  archivos utilizando el esquema de almacenamiento de Openshift ODF
  asociado a un esquema NFS, Administración de Personas, Consulta y
  transferencia de expedientes o partes de expedientes y anexos, etc
\end{enumerate}

\hypertarget{soluciuxf3n-2}{%
\subsection{Solución 2}\label{soluciuxf3n-2}}

\hypertarget{technology-service}{%
\subsubsection{Technology Service}\label{technology-service}}

\hypertarget{soluciuxf3n-1}{%
\subsection{Solución 1}\label{soluciuxf3n-1}}

\hypertarget{sint1.-integraciuxf3n.-ingreso-a-conti}{%
\subsubsection{SINT1. Integración. Ingreso a
Conti}\label{sint1.-integraciuxf3n.-ingreso-a-conti}}

Interoperabilidad IOP1. Transporte / Entrega Consulta Negocio\\
Modelo de datos (XML, RBDMS, \ldots) Esquema de datos (XSD, DTD,
JSON-E\ldots) Contratos de interoperabilidad (WSDL, API\ldots) Mensajes
petición IN (API, XML\ldots) Mensajes respuesta OUT (API, XML\ldots)
Mensajes excepción (API, XML\ldots) Transporte (REST, SOAP) Función
lógica (JEE, \ldots) Registro y envío de actividad

\hypertarget{sint2.-integraciuxf3n.-consulta-uxedtem-de-conti}{%
\subsubsection{SINT2. Integración. Consulta ítem de
Conti}\label{sint2.-integraciuxf3n.-consulta-uxedtem-de-conti}}

Interoperabilidad IOP1. Transporte / Entrega Consulta Negocio\\
Modelo de datos (XML, RBDMS, \ldots) Esquema de datos (XSD, DTD,
JSON-E\ldots) Contratos de interoperabilidad (WSDL, API\ldots) Mensajes
petición IN (API, XML\ldots) Mensajes respuesta OUT (API, XML\ldots)
Mensajes excepción (API, XML\ldots) Transporte (REST, SOAP) Función
lógica (JEE, \ldots) Registro y envío de actividad

\hypertarget{sint3.-integraciuxf3n.-radicar-uxedtem}{%
\subsubsection{SINT3. Integración. Radicar
ítem}\label{sint3.-integraciuxf3n.-radicar-uxedtem}}

Interoperabilidad IOP1. Transporte / Entrega Consulta Negocio\\
Modelo de datos (XML, RBDMS, \ldots) Esquema de datos (XSD, DTD,
JSON-E\ldots) Contratos de interoperabilidad (WSDL, API\ldots) Mensajes
petición IN (API, XML\ldots) Mensajes respuesta OUT (API, XML\ldots)
Mensajes excepción (API, XML\ldots) Transporte (REST, SOAP) Función
lógica (JEE, \ldots) Registro y envío de actividad

\hypertarget{sint4.-integraciuxf3n.-generaciuxf3n-de-documentos}{%
\subsubsection{SINT4. Integración. Generación de
documentos}\label{sint4.-integraciuxf3n.-generaciuxf3n-de-documentos}}

Interoperabilidad IOP1. Transporte / Entrega Consulta Negocio\\
Modelo de datos (XML, RBDMS, \ldots) Esquema de datos (XSD, DTD,
JSON-E\ldots) Contratos de interoperabilidad (WSDL, API\ldots) Mensajes
petición IN (API, XML\ldots) Mensajes respuesta OUT (API, XML\ldots)
Mensajes excepción (API, XML\ldots) Transporte (REST, SOAP) Función
lógica (JEE, \ldots) Registro y envío de actividad

\hypertarget{problema-1}{%
\subsection{Problema 1}\label{problema-1}}

\hypertarget{levantamiento}{%
\subsubsection{Levantamiento}\label{levantamiento}}

\hypertarget{contractual}{%
\subsubsection{Contractual}\label{contractual}}

\begin{enumerate}
\def\labelenumi{\arabic{enumi}.}
\tightlist
\item
  Implementación de al menos 20 servicios de Interoperabilidad bajo un
  esquema de Bolsa de Horas con una cantidad de 1.960. Las horas
  remanentes serán utilizadas en el desarrollo de servicios adicionales
\item
  Los al menos 20 servicios desarrollados serán entregados documentados
  y contarán con una garantía de seis meses.
\item
  Para la implementación de los servicios de interoperabilidad con
  entidades externas se utilizará el modelo XROAD definido por el
  Ministerio de Tecnologías de la Información y Comunicaciones para
  intercambio de información entre entidades del estado.
\end{enumerate}

\hypertarget{entrega}{%
\subsubsection{Entrega}\label{entrega}}

\hypertarget{reqr3.-integraciuxf3n-con-sistema-conti}{%
\subsection{REQR3. Integración con Sistema
Conti}\label{reqr3.-integraciuxf3n-con-sistema-conti}}

\ldots{}

\hypertarget{reqr2.-condiciones-tecnoluxf3gicas-jep}{%
\subsection{REQR2. Condiciones tecnológicas
JEP}\label{reqr2.-condiciones-tecnoluxf3gicas-jep}}

\ldots{}

Integraciones JEP, 2024

Organización de referencia. Integración JEP. Softgic. Servivcios,
Componentes, Roles de servicios.

versión 0.1.2

\hypertarget{contexto}{%
\section{1.contexto}\label{contexto}}

\begin{itemize}
\tightlist
\item
  \protect\hyperlink{Introducciuxf3n}{Introducción}
\item
  \protect\hyperlink{app:-mi-mutual-central-application-component}{app:
  Mi Mutual Central (Application Component)}

  \begin{itemize}
  \tightlist
  \item
    \protect\hyperlink{gestiuxf3n-usuarios-application-function}{Gestión
    Usuarios (Application Function)}
  \item
    \protect\hyperlink{gestiuxf3n-fondo-mutual-y-auxilio-funerario-application-function}{Gestión
    fondo mutual y auxilio funerario (Application Function)}
  \item
    \protect\hyperlink{configuraciuxf3n-factores-cuxe1lculos--contribuciones-application-function}{Configuración
    factores cálculos- contribuciones (Application Function)}
  \item
    \protect\hyperlink{interoperabilidad-entre-sistemas-coomeva-application-function}{Interoperabilidad
    entre sistemas Coomeva (Application Function)}
  \item
    \protect\hyperlink{gestiuxf3n-reclamaciones-application-function}{Gestión
    Reclamaciones (Application Function)}
  \item
    \protect\hyperlink{gestiuxf3n-beneficiarios-application-function}{Gestión
    Beneficiarios (Application Function)}
  \item
    \protect\hyperlink{administraciuxf3n-facturaciuxf3n-y-recaudo-application-function}{Administración
    facturación y recaudo (Application Function)}
  \item
    \protect\hyperlink{certificados-application-function}{Certificados
    (Application Function)}
  \item
    \protect\hyperlink{autorizaciones-application-function}{Autorizaciones
    (Application Function)}
  \item
    \protect\hyperlink{simuladores-application-function}{Simuladores
    (Application Function)}
  \item
    \protect\hyperlink{seguridad-application-function}{Seguridad
    (Application Function)}
  \end{itemize}
\item
  \protect\hyperlink{caracteruxedsticas-funcionales-requirement}{Características
  Funcionales (Requirement)}
\item
  \protect\hyperlink{restricciones-de-arquitectura-constraint}{Restricciones
  de Arquitectura (Constraint)}
\item
  \protect\hyperlink{autorizaciones-application-service}{Autorizaciones
  (Application Service)}
\item
  \protect\hyperlink{certificados-application-service}{Certificados
  (Application Service)}
\item
  \protect\hyperlink{configuraciuxf3n-application-service}{Configuración
  (Application Service)}
\item
  \protect\hyperlink{facturaciuxf3n-y-recaudo-application-service}{Facturación
  y Recaudo (Application Service)}
\item
  \protect\hyperlink{gestiuxf3n-de-beneficiarios-application-service}{Gestión
  de Beneficiarios (Application Service)}
\item
  \protect\hyperlink{gestiuxf3n-de-productos-application-service}{Gestión
  de Productos (Application Service)}
\item
  \protect\hyperlink{gestiuxf3n-de-reclamos-application-service}{Gestión
  de Reclamos (Application Service)}
\item
  \protect\hyperlink{gestiuxf3n-de-usuarios-application-service}{Gestión
  de Usuarios (Application Service)}
\item
  \protect\hyperlink{simuladores-application-service}{Simuladores
  (Application Service)}
\item
  \protect\hyperlink{unidad-de-solidaridad-y-seguros-business-function}{Unidad
  de Solidaridad y Seguros (Business Function)}
\end{itemize}

\hypertarget{introducciuxf3n-1}{%
\subsection{Introducción}\label{introducciuxf3n-1}}

\begin{figure}
\centering
\caption{1.contexto}
\end{figure}

El sistema principal de fondo Mi Mutual Central es la composición de las
funciones de negocio de la Unidad de Solidaridad de Coomeva. Las
funciones de negocio referidas, como Gestión Beneficiarios,
Certificados, Gestión Beneficiarios, aparecen dentro del componente
principal en la imagen.

Este entregable documenta los diferentes módulos y componentes que hacen
parte de la estructura de una aplicación en Angular 12 y como es su
interacción para conformar una arquitectura robusta y escalable para
aplicaciones de gran tamaño.

Las librerías Spring Boot Security y Spring Boot Oauth2 proveen
características de seguridad entre Vista (Angular 2) y Controlador.
Estas son responsables de que únicamente permita el acceso si se está
autenticado. Además, para realizar el proceso de autenticación se delega
a la aplicación SISPRO (Coomeva) que funciona como un servidor de
autenticación.

\end{document}
